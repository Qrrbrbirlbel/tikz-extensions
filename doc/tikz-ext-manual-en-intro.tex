% !TeX spellcheck = en_US
% !TeX root = tikz-ext-manual.tex
% Copyright 2022 by Qrrbrbirlbel
%
% This file may be distributed and/or modified
%
% 1. under the LaTeX Project Public License and/or
% 2. under the GNU Free Documentation License.
%
\part{Introduction}
\begin{multicols}{2}
\section{Usage}
This package is called |tikz-ext|, however,
one can't load it via |\usepackage|.%
\footnote{Except for \texttt{pgfcalendar-ext} and \texttt{pgffor-ext}.}
Instead, this package consists mostly of
\pgfname\space and \tikzname\space libraries
which are loaded by either |\usepgflibrary| or |\usetikzlibrary|.

\section{Why do we need it?}
Since I have been answering questions on
\hyperlink{https://tex.stackexchange.com}{TeX.sx}
I've noticed that some questions come up again and again,
every time with a slightly different approach on how to solve them.

I don't like reinventing the wheel which is why I've gathered
the solutions of my answers in this package.

\section{Having problems?}
Note however, that most of these extensions haven't been
stress-tested properly and might be considered
experimental.

Don't hesitate to open an issue on GitHub.
You probably found a bug.

\section{A word on namespaces and the introduction of \texttt{\textbackslash tikzextset}}
Since some parts of this package exist in some form since 2013,
the choice for key names and in which \pgfname keys namespace they reside
is not always optimal.
They often reside in the main |/tikz| or |/pgf| path.
Similar applies to macro names.

For future version, it is planned to move libraries in the |/tikz/ext| namespace.
For keys in the |/pgf| namespace, this will probably not happen
since it makes it not very intuitive to use them in \tikzname.

Starting from version 0.6,
\tikzextname\space provides the commands
\texttt{\textbackslash tikzextversion} and
\texttt{\textbackslash tikzextversionnumber}
for compatibility testing.
The latter simply increments with every release
so that the former doesn't need to be parsed.
\begin{command}{\tikzextversion}
  Returns \enquote{\tikzextversion}.
\end{command}
\begin{command}{\tikzextversionnumber}
  Returns \enquote{\tikzextversionnumber}.
\end{command}

Also starting from version 0.6, there's \texttt{\textbackslash tikzextset}.
\begin{command}{\tikzextset\marg{options}}
  This command will process the \meta{options} using the
  |\pgfkeys| command with the default path set to |/tikz/ext|.
\end{command}
\end{multicols}