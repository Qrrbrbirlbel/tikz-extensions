% !TeX TS-program = lualatex
% !TeX spellcheck = en_US
% !TeX root = tikz-ext-manual.tex
% Copyright 2023 by Qrrbrbirlbel
%
% This file may be distributed and/or modified
%
% 1. under the LaTeX Project Public License and/or
% 2. under the GNU Free Documentation License.
%

\section{Arrow Pics}
\label{tikzlibrary:arrows}
\tikzset{external/export/.try=false}%
\begin{tikzlibrary}{ext.arrows-plus}
  This library defines a few pics that can be used to place arrow tips on paths.
\end{tikzlibrary}

\begin{multicols}{2}
The \referenceDecorationandIndexO{markings} library already provides the functionality
to place arrow tips along the path.
The pics provided by this library serve as an alternative.

Many of the pics share various keys.
\begin{key}{/tikz/pos <=\meta{value} (initially 0.0)}
If the pic type supports it and an start arrow tip sequence is provided
this specifies the position of that sequence.
\end{key}
\begin{key}{/tikz/pos >=\meta{value} (initially 0.5)}
This is basically an alias for \referenceKeyandIndex{pos},
if an end arrow tip sequence is provided, it is placed at this position.
\end{key}

\begin{key}{/tikz/arrow shift mode=\meta{shift mode} (default total length)}
This key is used to set the \meta{shift mode} for the arrow tip.
It can be one of the following.
\begin{description}
\item[|arrow shift mode|=\declare{|off|}]

  This disables the shifting.
\item[|arrow shift mode|=\declare{|total length|}]

  The total length of the whole arrow tip sequence will be used.
\item[|arrow shift mode|=\declare{|total|}]

  This is an alias for |total length|.
\item[|arrow shift mode|=\declare{|length until line end|}]

  The length of the whole arrow tip until the line end will be used --
  as reported by \pgfname\ which might not always be the expected one.
\item[|arrow shift mode|=\declare{|line end|}]

  This is an alias for |length until line end|.
\end{description}

For single arrow tips it might be better to use the Centered arrow tip variants
of the \referenceLibraryandIndexExt{arrows} library (see sec~\ref{pgflibrary:arrows})
and disabled |arrow shift mode|.
\end{key}

When an arrow tip sequence is to be drawn depending on the shift mode
its total length or its length until the line end will be determined
and multiplied with the |arrow shift factor|.
The result if this evaluation is used to shift the arrow tip sequence
in the tip's direction.

\begin{key}{/tikz/arrow shift factor=\meta{value} (initially 0.5)}
  This determines the shift factor.
  
  The default value is probably good for most cases.
\end{key}
\begin{codeexample}[preamble=\usetikzlibrary{ext.arrows-plus}]
\begin{tikzpicture}[>={Straight Barb[color=red]}, ultra thick]
\ttfamily
\foreach[count=\y] \shiftmode in {off, total length, length until line end}
  \draw[arrow shift mode=\shiftmode] (0, -\y  )
                -- pic {arrow=>}    ++(right:2)
                -- pic {arrow=>.>>} ++(right:2) node[below right] {\shiftmode}
    ++(down:.4) -- pic {arrow=>.>>} ++( left:2)
                -- pic {arrow=>}    ++( left:2);
\draw[thin, gray] (1,-.75) -- +(down:3) (3,-.75) -- +(down:3);
\end{tikzpicture}
\end{codeexample}

\begin{pictype}{arrow}{\opt{|=|\meta{arrow tip specification}}}
  This pic draws the given \meta{arrow tip specification} (default |>|).
  
  This obviously is best use as a pic along a path segment that supports it.
  It \emph{does not} support bended arrow tips.
\begin{codeexample}[preamble=\usetikzlibrary{bending, ext.arrows-plus}]
\begin{tikzpicture}[>={Triangle[color=red]}, arrows={[bend]}, ultra thick]
\ttfamily
\foreach[count=\y] \shiftmode in {off, total length, length until line end}
  \draw[arrow shift mode=\shiftmode] (0, -\y  )
                to[bend  left] pic {arrow=>}    ++(right:2)
                to[bend  left] pic {arrow=>>.>} ++(right:2)
                                             node[below right] {\shiftmode}
    ++(down:.4) to[bend right] pic {arrow=>>.>} ++( left:2)
                to[bend right] pic {arrow=>}    ++( left:2);
\draw[thin, gray] (1,-.75) -- +(down:3) (3,-.75) -- +(down:3);
\end{tikzpicture}
\end{codeexample}
\end{pictype}

The previous |arrow| pic can be used with arcs but for bended arrow tips,
the following pic is needed.
\begin{pictype}{arc arrows}{\opt{|=|\meta{start tip specification}|-|\meta{end tip specification}}}
  This pic draws the given tips along an arc. It supports the |pos <| key.
  
  The shift mode applies to both ends.
  
  Since arcs don't have a target coordinate a |pos| or a |pos >| needs to be specified explicitly.
\begin{codeexample}[preamble=\usetikzlibrary{bending, ext.arrows-plus}]
\begin{tikzpicture}[>={Stealth[color=red, round]}, arrows={[bend]}, ultra thick]
\ttfamily
\foreach[count=\y] \shiftmode in {off, total length, length until line end}
  \draw[arrow shift mode=\shiftmode, x radius=2, y radius=3] (0, -\y  )
                arc[start angle=120, delta angle=-60] pic[midway] {arc arrow=>}
                arc[start angle=120, delta angle=-60] pic[midway] {arc arrow=>.>>}
                                             node[below right] {\shiftmode}
    ++(down:.4) arc[end angle=120, delta angle=60] pic[midway] {arc arrow=>.>>}
                arc[end angle=120, delta angle=60] pic[midway] {arc arrow=>};
\draw[thin, gray] (1,-.75) -- +(down:3) (3,-.75) -- +(down:3);
\end{tikzpicture}
\end{codeexample}

\paragraph{Tip:}
Use an arc with the full 360${}^\circ$ to place bended arrow tips along a circle or an ellipse.
\paragraph{Tip:}
Use an empty option specification |[]| after the first arrow tip in the start specification
to help it correctly apply the shifting distance.
\begin{codeexample}[preamble=\usetikzlibrary{bending, ext.arrows-plus}]
\begin{tikzpicture}[>={Stealth[color=red, round]}, arrows={[bend]}, ultra thick]
\draw[start angle=0, end angle=360, radius=1]
  (0, 0) arc[] pic[pos <=.25, pos >=.75] {arc arrows=<<<->>>}   -- cycle
  (3, 0) arc[] pic[pos <=.25, pos >=.75] {arc arrows=<[]<<->>>} -- cycle;
\end{tikzpicture}
\end{codeexample}
\end{pictype}
\end{multicols}
\endinput