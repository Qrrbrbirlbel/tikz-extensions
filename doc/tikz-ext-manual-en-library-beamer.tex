% !TeX TS-program = lualatex
% !TeX spellcheck = en_US
% !TeX root = tikz-ext-manual.tex
% Copyright 2025 by Qrrbrbirlbel
%
% This file may be distributed and/or modified
%
% 1. under the LaTeX Project Public License and/or
% 2. under the GNU Free Documentation License.
%

\section{Beamer with \tikzname}
\label{tikzlibrary:beamer}
\begin{tikzlibrary}{ext.beamer}
  This library can help create \tikzname\ diagrams in the class Beamer\indexPackageO{beamer}.
\end{tikzlibrary}

%\begin{multicols*}{2}
\subsection{Helpers}
These helpers are always available, even if this library is loaded outside of Beamer.
\begin{key}{/\tikzext/ignore line width}
If this key is used on a scope (or the \tikzname\ picture itself),
the line widths of paths will not contribute to the bounding box
of the diagram.
\end{key}
\begin{key}{/\tikzext/max bounding box=\meta{name}}
This key is to be used on multiple |tikzpicture| environments.
All \tikzname\ diagram with the same \meta{name} will have the same bounding box.
Refrain from using (unprotected) commas (|,|) in the \meta{name}.

This uses the \filetype{aux} file and
is therefore incompatible with the \referenceLibraryandIndexO{external} library.

However, it is made compatible with the
\referencePackageandIndexO{memoize} \cite{memoize} package,
even if it takes a few compilations until it is stable again
after a new diagram is added to the group.
\end{key}

\subsection{Beamer}
While \tikzname\ has some rudimentary support for the |beamer| class,
i.\,e. in the form of \texttt{\string\path<}\meta{overlay specification}\texttt{>},
this uses Beamer's |\alt| command internally so that on overlays
that are not included in \meta{overlay specification}, the path will not be typeset
and will therefore not contribute to the diagram's bounding box.

This in turn will lead to the diagram \enquote{jumping around}
as every overlay will contain a different diagram with different dimensions.
The \referencePackageandIndexO{aobs-tikz} package solves this by setting the |opacity| to zero
for all those slides an element shouldn't be visisble on.

I believe we can do better.

Though, remember that for many simple diagrams,
you can simply use \referenceCommandandIndexO{\onslide}.
The following diagram will show
\begin{itemize}
\item nodes and edges transparent on overlay 1,
\item nodes fully visible and edges transparent on overlay 2 and
\item all elements fully visible on overlay 3.
\end{itemize}
\begin{codeexample}[preamble=\usetikzlibrary{ext.beamer} \setbeamercovered{transparent},code only]
\begin{tikzpicture}
\onslide<2->
\path node (a)          {A}
      node (b) at (1,2) {B};
\onslide<3->
\path (a) edge[bend right] (b);
\onslide
\end{tikzpicture}
\end{codeexample}
\subsubsection{Stop Jumping}
One solution to this is to have the same \tikzname\ diagram
have the same size on every overlay.
The |/tikz/ext/ignore line width| might help if all that changes between overlays
is the line width of elements.

Another one is the following key.
\begin{key}{/\tikzext/sync bounding box}
This key uses |ext/max bounding box| with a specific \meta{name}
that is stable across overlays.

If you find yourself often rearrange diagrams or changing overlays,
you might be better off using the |ext/sync bounding box| key
directly with a distinct \meta{name}.
\end{key}

\subsubsection{Beamer Function and keys}
\begin{key}{/\tikzext/beamer function=\mchoice{original, alt, only, uncover, visible, invisible}}
This key changes how the \meta{overlay specification} in
\texttt{\string\path<}\meta{overlay specification}\texttt{>}
is applied internally.
The choices |original|, |alt| and |only| are all the same
and will result in the default behavior of \tikzname.

The same overlays as above can be created now with the following diagram.
\begin{codeexample}[preamble=\usetikzlibrary{ext.beamer} \setbeamercovered{transparent},code only]
\begin{tikzpicture}[ext/beamer function=uncover]
\path<2-> node (a)          {A}
          node (b) at (1,2) {B};
\path<3-> (a) edge[bend right] (b);
\end{tikzpicture}
\end{codeexample}
\end{key}

\begin{keylist}[/\tikzext]{%
    uncover=\meta{overlay specification} (default all),
      cover=\meta{overlay specification} (default all),
    visible=\meta{overlay specification} (default all),
  invisible=\meta{overlay specification} (default all)%
}
These keys work similar to Beamer's own |\onslide| command
but only apply to the element it is used on.

The implementation of this is rather experimental and should be used carefully.
Multiple uses of these options \emph{don't} stack, only the last one wins --
this includes the use of the first syntax shorcuts introduced
in the next subsubsection.

The same overlays as above can be created with the first of the following diagram.
In the second diagram |ext/uncover| is only used on the (actual) empty path.
Just like |draw|/|\draw|, the nodes will not observe the request to be covered on overlay 1.
\begin{codeexample}[preamble=\usetikzlibrary{ext.beamer} \setbeamercovered{transparent},code only]
\begin{tikzpicture}
\path[nodes={ext/uncover=2-}]
  node (a)          {A}
  node (b) at (1,2) {B}
  (a) edge[bend right, ext/uncover=3-] (b);
\end{tikzpicture}
\begin{tikzpicture}
\draw[ext/uncover=2-]
  node (a)          {A}
  node (b) at (1,2) {B};
\end{tikzpicture}
\end{codeexample}
\end{keylist}
\begin{keylist}[/\tikzext]{%
  aobs   visible=\meta{overlay specifcation} (default all),
  aobs invisible=\meta{overlay specifcation} (default all)%
}
In case one wants to use the method of simply setting the opacity of elements to zero
to hide them, these keys are also available.
\end{keylist}
Of course, an extension to Beamer is not complete without the following keys.
\begin{key}{/\utilsext/only=\marg{overlay specification}\marg{key-value list}}
  Applies the \meta{key-value list} only on \meta{overlay specification}.
\end{key}
\begin{key}{/\utilsext/alt=\marg{overlay specification}\marg{default kv list}\marg{alternative kv list}}
  Applies the \meta{default kv list} on \meta{overlay specification},
  otherwise the \meta{alternative kv list}.
\end{key}
\begin{key}{/\utilsext/temporal=\marg{overlay specification}%
  \marg{before kv list}\marg{default kv list}\marg{after kv list}}
  Applies the specific list depending whether the current overlay is before,
  on or after the specified \meta{overlay specification}.
\end{key}
\subsubsection{Beamer Shortcuts}
But, of course, no one wants to write |/utils/ext/only={2}{red}| to
make an element red on overlay 2.
\begin{key}{/\tikzext/beamer shortcuts=\marg{key-value list}}
  This executes the \meta{key-value list} in the namespace |/tikz/ext/beamer shortcuts|.
\end{key}
\begin{key}{/\tikzext/beamer shortcuts/aot}
  This forwards the keys \referenceKeyandIndex{alt}, \referenceKeyandIndex{only}
  and \referenceKeyandIndex{temporal} to the aforementioned homonymous |/utils/ext| keys.
\end{key}
\begin{key}{/\tikzext/beamer shortcuts/first char=%
  \mchoice{uncover, cover, visible, invisible, aobs visible, aobs invisible} (initially uncover)}
  The value of this key will be used for the first char shorthands that can be enabled with the following keys.
\end{key}
\begin{key}{/\tikzext/beamer shortcuts/enable first char <}
  This install a \enquote{first char} handler with the character |<|.
  
  This allows the example diagram to specified in the following way.
\begin{codeexample}[preamble=\usetikzlibrary{ext.beamer} \setbeamercovered{transparent},code only]
\begin{tikzpicture}[ext/beamer shortcuts={enable first char <}]
\node (a) [<2->]          {A};
\node (b) [<2->] at (1,2) {B};
\path (a) edge[<3->, bend right] (b);
\end{tikzpicture}
\end{codeexample}

  Internally, this will be converted to |ext/uncover=|\marg{overlay specification}.
  
  Actually, the full syntax is much more versatile
  \begin{quote}
      |<|\meta{overlay specification}|>|\opt{|'|}\opt{\meta{options}}\\
      |<|\meta{overlay specification}|>|\opt{|'|}\opt{\marg{options A}}\opt{\marg{options B}}\opt{\marg{options C}}
  \end{quote}
  
  If no options are present the \meta{overlay specification} will
  be forwarded to one of the keys explained in the subsection above
  -- depending on |ext/beamer shortcuts/first char|.
  The optional |'| after |>| will invert the \meta{overlay specification}.
  
  If \meta{options} or \marg{options A} are present,
  these will only applied on \meta{overlay specifications} (or the inverse of them with the |'|).
  If two sets of options are present, they will be |\alt|ed, three options will be |\temporal|ed.
  
  The |'| will swap two sets of options while for three the first and the last will be swapped.
\end{key}
\begin{key}{/\tikzext/beamer shortcuts/enable first char=\marg{character}}
  As the |<| character might lead to problems as it conflicts with the \tikzname\ shorthand
  of specifying arrow tip sequences
  (i.\,e. the famous |<->|\footnote{Though, remember, you can always write \texttt{arrows = <->}.})
  and the \referenceLibraryandIndexO{graphs} library's own first char syntax
  an alternative is presented here.
  
  This key enables a first char syntax with \meta{character} where the full syntax is the same as above:
  \begin{quote}
      \meta{character}|<|\meta{overlay specification}|>|\opt{|'|}\opt{\meta{options}}\\
      \meta{character}|<|\meta{overlay specification}|>|\opt{|'|}\opt{\marg{options A}}\opt{\marg{options B}}\opt{\marg{options C}}
  \end{quote}
  
  This means, the example diagram can be created in the following way.
\begin{codeexample}[preamble=\usetikzlibrary{ext.beamer} \setbeamercovered{transparent},code only]
\begin{tikzpicture}[ext/beamer shortcuts={enable first char=?}]
\node (a) [?<2->]          {A};
\node (b) [?<2->] at (1,2) {B};
\path (a) edge[?<3->, bend right] (b);
\end{tikzpicture}
\end{codeexample}
\end{key}

\subsubsection{Key Handler}
Maybe this is a syntax that someone wants \dots
\begin{ext_handler}{{<}|=|\meta{overlay specification}|>|\opt{value}}
  This handler applies the key on \meta{overlay specification}.
  If \meta{value} is missing, then the key is also used without a value.
  For an empty value, use |{}|.
  \begin{key}{/\tikzext/beamer shortcuts/enable handler}
    If |ext_| is too much, using this key activates the |.<| handler.
    \begin{handler}{{<}|=|\meta{overlay specification}|>|\opt{value}}
      This handler is then an alias for the |.ext_<| handler.
    \end{handler}
  \end{key}
\end{ext_handler}
%\end{multicols*}
\endinput