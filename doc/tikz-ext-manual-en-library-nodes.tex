% !TeX root = tikz-ext-manual.tex
% !TeX spellcheck = en_US
% Copyright 2022 by Qrrbrbirlbel
%
% This file may be distributed and/or modified
%
% 1. under the LaTeX Project Public License and/or
% 2. under the GNU Free Documentation License.
%

\section{Nodes}
\begin{tikzlibrary}{ext.nodes}
\end{tikzlibrary}

\begin{stylekey}{/tikz/node on line=\opt{\meta{anchor specification}} (default |\{\}|)}
This installs a \referenceKeyandIndexO{to path} that places \emph{one}
node along a straight line but connect the line with it.

This allows a node to be placed \emph{on} a straight line without having to
use |fill = white| or similar tricks to make the line disappear 
beneath the node.

The optional \meta{anchor specification} allows to specify the
anchors to which the line should connect.
It allows one or two anchors divided by | and | to be specified.
\end{stylekey}

\begin{stylekey}{/tikz/nodes on line}
This is similar to the previous key but allows
multiple nodes to be placed on a straight line
\emph{if} they are in the correct order (from start to target),
don't overlap with each other, the start or the target.

It allows \emph{no} anchor specification.
\end{stylekey}

\begin{codeexample}[preamble=\usetikzlibrary{ext.nodes, quotes}, width=7cm]
\tikz[inner sep=.15em, circle, nodes=draw, sloped]
  \draw[ultra thick, ->, node on line] (0,0) to["0"] (1,1)
                                             to["1"] (2,0)
          to[nodes on line, "2.1" near start, "2.2", "2.3" near end] (5,1);
\end{codeexample}
\begin{codeexample}[preamble=\usetikzlibrary{ext.nodes, quotes}, width=7cm]
\tikz[inner sep=.15em, nodes=draw]
  \draw[thick, ->, node on line=west and east] (0,0) to["0"] (1,1)
                                                     to["1"] (2,0)
                                                     to["2"] (4,1);
\end{codeexample}

The following keys need the \referenceLibraryandIndexO{intersections}
and the \referenceLibraryandIndexExt{spath3} \cite{spath3}
library to be loaded. They will not be automatically
loaded by this library.

\begin{stylekey}{/tikz/nodes on curve=\meta{to path} (default line to)}
Similar to \referenceKeyandIndex{nodes on line}, this key allows
to have nodes on arbitrary paths
\end{stylekey}

\begin{stylekey}{/tikz/nodes on curve'=\meta{to path} (default line to)}
\end{stylekey}

\begin{codeexample}[preamble=\usetikzlibrary{ext.nodes, intersections, quotes, spath3}, width=7cm]
\begin{tikzpicture}[ultra thick]
  \node (A) at (0, 0) {A} ;
  \node (B) at (3, 0) {B} ;
  \draw [red, ->, nodes on curve'=bend left]
    (A) to node[blue,draw]{label} (B)
        to ["X" {sloped, near start},
            "Z" {sloped, near end},
            "Y"] (A);
\end{tikzpicture}
\end{codeexample}
\begin{codeexample}[preamble=\usetikzlibrary{ext.nodes, intersections, quotes, spath3}, width=7cm]
\tikz[inner sep=.15em, circle, nodes={draw, green}, sloped, ultra thick]
  \draw[->, nodes on curve=bend left] (0,0) to["0"] (1,1)
                                            to["1"] (2,0)
              to["2" near start, "3", "4" near end] (4,1)
                                            -- ++(down:1);
\end{codeexample}
