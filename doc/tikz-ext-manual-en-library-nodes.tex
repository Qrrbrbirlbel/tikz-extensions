% !TeX root = tikz-ext-manual.tex
% !TeX spellcheck = en_US
% Copyright 2022 by Qrrbrbirlbel
%
% This file may be distributed and/or modified
%
% 1. under the LaTeX Project Public License and/or
% 2. under the GNU Free Documentation License.
%

\section{Nodes}
\begin{tikzlibrary}{ext.nodes}
This library extends \tikzname's functionalities when it comes to nodes.
\inspiration{NodesOnLine-Q, NodesOnCurve-Q}{NodesOnLine-A, NodesOnCurve-A}
\end{tikzlibrary}

\begin{multicols}{2}
\subsection{Pic as a node}
\begin{key}{/\tikzext/pic=\opt{\meta{boolean}} (default true, initially false)}
\keycompat{tikz}
This key allows one to use a pic where usually only nodes are accepted,
for example as a label.
\begin{codeexample}[preamble=\usetikzlibrary{ext.nodes, ext.misc}]
\begin{tikzpicture}[
  slsl/.pic={\draw(-2pt, 1.5pt)--( 2pt, .5pt)
                  ( 2pt,-1.5pt)--(-2pt,-.5pt);}]
\node[
  draw, minimum width=3cm, minimum height=1cm,
  label={[ext/pic            ] east:slsl},
  label={[ext/pic, rotate= 90]north:slsl},
  label={[ext/pic            ] west:slsl},
  label={[ext/pic, rotate=-90]south:slsl}]{};
\end{tikzpicture}
\end{codeexample}
\end{key}
\newcolumn

\subsection{Node as a pic}
A pic allows any (sub)picture to be handled like a node --
with one caveat: it doesn't provide anchors like a node and
is thus not able to be placed like one.

A \pgfname\ matrix on the other hand allows a (sub)picture in a node (multiple even)
but the matrix is not able to be transformed.

The following should find a compromise between these two.
Here are some rules and guidelines for using this:

\begin{itemize}
\item The \enquote{text} of a node is a (sub)picture like a pic's code or a matrix cell.
      The (sub)picture will be typeset as the text of the node.
\item Only the |text| node part is available. The |\nodepart| command should not be used.
      The solution is \emph{not} a matrix. It can only contain one cell. No |\\| is needed at the end.
\item The baseline of the node is the (sub)pictures $y$ value of $0$.
\item Just like with a pic or a matrix, certain settings and values will be inherited by the (sub)picture.
\item The bounding box of the (sub)picture will be evaluated automatically and the result will be used to determine
      the size of the \enquote{text}.

      However, the line width will not contribute to the size.
      This has the advantage that with |/pgf/inner sep| set to zero, paths of the (sub)picture
      can lie on the border of the node.
\item This is not nestable!
      You cannot use |node is pic| inside a (sub)picture created by |node is pic|.
\end{itemize}

\newcolumn
\begin{key}{/\tikzext/node is pic=\opt{\meta{true or false}} (default true)}
  Setting this to |true| activates the aforementioned procedure for the node's text to be a picture.
  Setting it to |false| returns to the default behavior of a normal node.
\end{key}
\begin{stylekey}{/\tikzext/every node pic}
This key will be applied at the start of every (sub)picture.
\end{stylekey}
\begin{key}{/\tikzext/node pic reset graphic state}
This key sets up |/tikz/ext/every node pic| so that it resets many settings and values
to their \pgfname/\tikzname\ default.
\end{key}

\begin{key}{/\tikzext/name prefix ..}
  Similar to a pic, |/tikz/name prefix| is set to the name of the outer node.
  This key will return to the name prefix that was set beforehand.
\end{key}
\begin{codeexample}[preamble=\usetikzlibrary{ext.nodes}]
\begin{tikzpicture}[x=+7mm, y=+7mm]
\draw (0,0) to[bend left]
  node[ext/node is pic, sloped, draw, above]{
    \draw[red] (0,0) circle[radius=1];
  } (3,3);
\end{tikzpicture}
\end{codeexample}
\newcolumn

\subsection{Nodes on paths}
When nodes are placed along paths they don't interrupt
the path at that place.
The decoration \referenceLibraryandIndexO{markings}
and its \referenceKeyandIndexO[/pgf/decoration/]{mark connection node}
key can help but only works for straight paths and
doesn't play nicely with arrow tips.

This library provides alternatives.
These are separated into straight paths, i.\,e. \referencePathOperationandIndexO{--},
and everything else (including any |to path|).

\subsubsection{Nodes on Lines}

\begin{stylekey}{/\tikzext/node on line=\opt{\meta{anchor specification}} (default |\{\}|)}
\keycompat{tikz}
This installs a \referenceKeyandIndexO{to path} that places \emph{one}
node along a straight line but connect the line with it.

This allows a node to be placed \emph{on} a straight line without having to
use |fill = white| or similar tricks to make the line disappear 
beneath the node.

The optional \meta{anchor specification} allows to specify the
anchors to which the line should connect.
It allows one or two anchors divided by | and | to be specified.
\end{stylekey}

\begin{stylekey}{/\tikzext/nodes on line}
\keycompat{tikz}
This is similar to the previous key but allows
multiple nodes to be placed on a straight line
\emph{if} they are in the correct order (from start to target),
don't overlap with each other, the start or the target.

It allows \emph{no} anchor specification.
\end{stylekey}

\begin{codeexample}[preamble=\usetikzlibrary{ext.nodes, quotes}]
\tikz[inner sep=.15em, circle, nodes=draw, sloped]
  \draw[ultra thick, ->, ext/node on line] (0,0) to["0"] (1,1)
                                                 to["1"] (2,0)
    to[ext/nodes on line, "2.1" near start, "2.2", "2.3" near end] (5,1);
\end{codeexample}
\begin{codeexample}[preamble=\usetikzlibrary{ext.nodes, quotes}]
\tikz[inner sep=.15em, nodes=draw]
  \draw[thick, ->, ext/node on line=west and east]
     (0,0) to["0"] (1,1)
           to["1"] (2,0)
           to["2"] (4,1);
\end{codeexample}

\subsubsection{Nodes on Curves}
The following keys need the \referenceLibraryandIndexO{intersections}
and the \referenceLibraryandIndexExt{spath3} \cite{spath3}
library to be loaded. They will not be automatically
loaded by this library.

Any \referenceKeyandIndexO[/pgf/]{outer sep} will be ignored.

If you can, use \texttt{fill=\meta{bg color}}
instead of these keys, it will be much faster and easier.

\begin{stylekey}{/\tikzext/nodes on curve=\meta{to path} (default line to)}
\keycompat{tikz}
Similar to |nodes on line|, this key allows
to have nodes on arbitrary paths.

This is not suitable for paths connecting nodes.
\end{stylekey}

\begin{stylekey}{/\tikzext/nodes on curve'=\meta{to path} (default line to)}
\keycompat{tikz}
As above but suitable for connecting nodes.
\end{stylekey}

\begin{codeexample}[preamble=\usetikzlibrary{ext.nodes, intersections, quotes, spath3}]
\begin{tikzpicture}[ultra thick]
  \node (A) at (0, 0) {A} ;
  \node (B) at (3, 0) {B} ;
  \draw [red, ->, ext/nodes on curve'=bend left]
    (A) to node[blue,draw]{label} (B)
        to ["X" {sloped, near start},
            "Z" {sloped, near end},
            "Y"] (A);
\end{tikzpicture}
\end{codeexample}
\begin{codeexample}[preamble=\usetikzlibrary{ext.nodes, intersections, quotes, spath3}]
\tikz[inner sep=.15em, circle, nodes={draw, green}, sloped, ultra thick]
  \draw[->, ext/nodes on curve=bend left] (0,0) to["0"] (1,1)
                                                to["1"] (2,0)
                  to["2" near start, "3", "4" near end] (4,1)
                                                -- ++(down:1);
\end{codeexample}
\newcolumn

\subsection{Automatic placement of nodes}
The \referenceKeyandIndexO{auto} key allows automatic placement of
nodes along a path segment.
This library extends this in various ways.

\subsubsection{More than left and right}
Besides |left| and |right| that are provided by \tikzname\ 
the following placement mechanism are provided:
\begin{itemize}
\item |ext/left| will place a node to the left of the direction of the line,
\item |ext/right| will place a node to the right of the direction of the line,
\item |ext/above| will place a node towards the direction of the line,
\item |ext/below| will place a node against the direction of the line,
\item |ext/west| will place a node towards the left side of the paper,
\item |ext/east| will place a node towards the right side of the paper,
\item |ext/north| will place a node towards the upper side of the paper and
\item |ext/south| will place a node twoards the lower side of the paper.
\end{itemize}
The placement mechanisms |ext/left| and |ext/right| are like the original
|left| and |right| mechanisms but don't swap sides when \referenceKeyandIndexO{sloped}
is used.

Certain cases exist for |ext/west|, |ext/east|, |ext/north| and |ext/south| placements
where it is not clear how a node should be placed.
These cases and their behavior can be seen in the following figure.%~\ref{fig:autoplacements}.
%\begin{figure*}
\begin{center}
\small
\newcommand*\tikzauto[3]{%
  \path[{ext_Centered Circle[black]}-Latex]
    (0,0) edge[auto=ext/#2] node {#2} (left:2) edge[auto=ext/#2] node {#2} (right:2)
          edge[auto=ext/#3] node {#3} (  up:2) edge[auto=ext/#3] node {#3} ( down:2);}
\begin{tikzpicture}[x=+.75cm,y=+.75cm, nodes=draw]
\path[{ext_Centered Circle[black]}-Latex]
    (0,0) edge node[auto=ext/west] {west}
               node[auto=ext/east] {east} (left:2)
          edge node[auto=ext/west] {west}
               node[auto=ext/east] {east} (right:2)
          edge node[auto=ext/north] {north}
               node[auto=ext/south] {south} (  up:2)
          edge node[auto=ext/north] {north}
               node[auto=ext/south] {south} (down:2);
\end{tikzpicture}
\end{center}
%\caption{Behavior of \texttt{ext/est}, \texttt{ext/east}, \texttt{ext/north} and \texttt{ext/south} in certain cases}
%\label{fig:autoplacements}
%\end{figure*}

\subsubsection{Offset}
Nodes are usually placed with their border (including any |outer sep|)
on the line.
With the following option, a node will be shifted a certain offset distance.

\begin{key}{/\tikzext/auto with offset=\opt{\meta{true or false}} (default true)}
  This key activates the offset function.
\end{key}
\begin{key}{/\tikzext/auto offset (initially 1cm)}
The offset distance itself.
\end{key}

For the \referenceDecorationandIndexO{brace} decoration,
the following keys are provided which needs the
\referenceLibraryandIndexO{decorations.pathreplacing} loaded before they
can be used.
\begin{key}{/\tikzext/nodes/install auto offset for brace decoration=\opt{\meta{distance}} (default 0pt)}
This key installs the necessary customizations
for the \referenceKeyandIndexO[/pgf/decoration/]{raise} key
so that the given value is available as an offset.

It also makes available the following keys.
\begin{key}{/\tikzext/auto offset for brace decoration}
This sets \referenceKeyandIndex[/\tikzext/]{auto offset} to
\texttt{\textbackslash pgfdecorationsegmentamplitude+
  (\textbackslash pgfkeysvalueof\{/pgf/decoration/raise\})}.
\end{key}
%\columnbreak
\begin{stylekey}{/\tikzext/every brace node}
Using this key on a node along a path that's decorated by the |brace| decoration
will offset the node so that it will be placed at the tip of the brace.
\end{stylekey}
\end{key}

\paragraph{Implementation note:}
This redefines the keys \referenceKeyandIndexO{auto},
\referenceKeyandIndexO{swap} and \referenceKeyandIndexO{sloped}.
One can install custom auto placement rules by using the following key.
\begin{key}{/\tikzext/nodes/install auto=\marg{left}\marg{right}}
  This key defines \texttt{/tikz/auto/\meta{left}} which activates the auto placement
  and installs the appropriate placement function.
  Further more, the key \texttt{/tikz/swap/\meta{left}} will be defined
  to active the \meta{right} placement function.
  
  The key |/tikz/swap| has been defined to apply \texttt{/tikz/swap/\meta{dir}}
  where \meta{dir} is the current placement function.
\end{key}
\subsubsection{Precise placement}
The default behavior of the |auto| placement mechanism is
to snap to one of the eight compass directions.
\begin{key}{/\tikzext/precise auto angle=\opt{\meta{true or false}} (default true)}
With this option set to |true|, the |auto| placement won't snap
to one of the eight compass directions.

This key disables the \referenceKeyandIndexO{sloped} option
which in turn will disable this option.
\end{key}
\newcolumn
\null
\end{multicols}