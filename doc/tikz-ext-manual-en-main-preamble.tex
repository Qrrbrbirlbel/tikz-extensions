% !TeX spellcheck = <none>
% !TeX root = tikz-ext-manual.tex
% Copyright 2022 by Qrrbrbirlbel
%
% This file may be distributed and/or modified
%
% 1. under the LaTeX Project Public License and/or
% 2. under the GNU Free Documentation License.
%

\usepackage[version=latest]{pgf}

\usepackage{xkeyval,calc,listings,tikz,fp}

\usepackage{imakeidx}
\makeindex

%%% Bibliography
\usepackage{csquotes}
\usepackage[backend=biber,style=numeric,backref]{biblatex}
\addbibresource{tikz-ext-manual.bib}

\usepackage{hyperref}
\hypersetup{%
    colorlinks=true, % use true to enable colors below:
    linkcolor=blue,%red,
    filecolor=blue,%magenta,
    urlcolor=blue,%cyan,
    citecolor=blue,
    pdfborder=0 0 0,
}

\usepackage[a4paper,left=2.25cm,right=2.25cm,top=2.5cm,bottom=2.5cm,nohead,columnsep=1cm]{geometry}
\usepackage{amsmath,amssymb}
\usepackage{xxcolor}
\usepackage{pifont}

\usepackage[T1]{fontenc}
% Promote `Missing character` reports to full errors
% require texlive 2021 or above
\tracinglostchars=3

\input{pgfmanual-en-macros}
%\newenvironment{pgfkeyslibrary}[1]{% from pgfmanual-en-macros
%  \begin{pgfmanualentry}%
%    \pgfmanualentryheadline{%
%      \pgfmanualpdflabel{#1}{}%
%      \textbf{pgfkeys Library} \texttt{\declare{#1}}}%
%    \index{#1@\protect\texttt{#1} pgfkeys library}%
%    \index{pgfkeys Libraries!#1@\protect\texttt{#1}}%
%    \vskip.25em
%    {{\ttfamily\char`\\usepgfkeyslibrary\char`\{\declare{#1}\char`\}\space\space \char`\%\space\space  \LaTeX\space and plain \TeX}}\\
%    {{\ttfamily\char`\\usepgfkeyslibrary[\declare{#1}]\space \char`\%\space\space Con\TeX t}}\\[.5em]%
%    \pgfmanualbody
%}%
%{%
%  \end{pgfmanualentry}%
%}

\usepackage{unicode-math}
\setmathfont[Scale=MatchUppercase]{libertinusmath-regular.otf}
\usepackage[ttscale=.85]{libertine}
\setmonofont[Scale=0.8]{Bitstream Vera Sans Mono}

\makeatletter
\renewcommand*\l@section[2]{%
  \ifnum \c@tocdepth >\z@
    \addpenalty\@secpenalty
    \addvspace{1.0em \@plus\p@}%
    \setlength\@tempdima{2.5em}%
    \begingroup
      \parindent \z@ \rightskip \@pnumwidth
      \parfillskip -\@pnumwidth
      \leavevmode \bfseries
      \advance\leftskip\@tempdima
      \hskip -\leftskip
      #1\nobreak\hfil \nobreak\hb@xt@\@pnumwidth{\hss #2}\par
    \endgroup
  \fi}
\renewcommand*\l@subsection{\@dottedtocline{2}{2.5em}{3.3em}}
\renewcommand*\l@subsubsection{\@dottedtocline{3}{5.8em}{4.2em}}
\def\@pnumwidth{2.2em}
\makeatother

% Global styles:
\tikzset{
  every plot/.style={prefix=plots/pgf-},
  shape example/.style={
    color=black!30,
    draw,
    fill=yellow!30,
    line width=.5cm,
    inner xsep=2.5cm,
    inner ysep=0.5cm}
}

% Detect changed labels
% by David Carlisle https://tex.stackexchange.com/a/169245
\makeatletter
\def\@testdef#1#2#3{%
  \def\reserved@a{#3}%
  \expandafter\ifx\csname #1@#2\endcsname\reserved@a\else
    \typeout{^^JLabel `#2' changed:^^J\meaning\reserved@a^^J\expandafter\meaning\csname #1@#2\endcsname^^J}%
    \@tempswatrue
  \fi}
\makeatother

\usetikzlibrary{
  ext.arrows,
  ext.calendar-plus,
  ext.misc,
  ext.node-families,
  ext.node-families.shapes.geometric,
  ext.nodes,
  ext.paths.arcto,
  ext.paths.ortho,
  ext.paths.timer,
  ext.patterns.images,
  ext.positioning-plus,
  ext.scalepicture,
  ext.shapes.heatmark,
  ext.shapes.circlearrow,
  ext.shapes.circlecrosssplit,
  ext.shapes.rectangleroundedcorners,
  ext.shapes.superellipse,
  ext.shapes.uncenteredrectangle,
  ext.topaths.arcthrough,
  ext.transformations.mirror,
  %
  calc,
  cd,
  fit,
  intersections,
  matrix,
  shapes.geometric,
  spath3,
  through,
  trees,
}
\usepackage{pgffor-ext}
%% needs Lua!
\usetikzlibrary{graphs,graphdrawing}
\usegdlibrary{force}

\makeatletter
\def\endofcodeexample#1{% original from tex/latex/pgf/doc/pgfmanual-en-macros.tex by Till Tantau et al
  \endgroup%
  \ifpgfmanual@setup@code%
    \gdef\pgfmanual@do@this{%
      {%
        \returntospace%
        \commenthandler%
        \xdef\code@temp{#1}% removes returns and comments
      }%
      \edef\pgfmanualmcatcode{\the\catcode`\^^M}%
      \catcode`\^^M=9\relax%
      \expandafter\scantokens\expandafter{\code@temp}%
      \catcode`\^^M=\pgfmanualmcatcode%
    }%
  \fi%
  \ifcode@hidden\else
    \ifcode@execute%
      \setbox\codeexamplebox=\hbox{%
        \ifx\code@render\pgfutil@empty%
        {%
          {%
            \returntospace%
            \commenthandler%
            \xdef\code@temp{#1}% removes returns and comments
          }%
          \catcode`\^^M=9%
          \colorbox{graphicbackground}{\color{black}\ignorespaces%
            \code@pre\expandafter\scantokens\expandafter{\code@temp\ignorespaces}\code@post\ignorespaces}%
        }%
        \else%
          \global\let\code@temp\code@render%
          \colorbox{graphicbackground}{\color{black}\ignorespaces%
            \code@render}%
        \fi%
      }%
      \ifx\code@animation@list\pgfutil@empty%
      \else%
      \setbox\codeexampleboxanim=\vbox{%
        \rightskip0pt\leftskip0pt plus1filll%
        \ifdim\wd\codeexamplebox>\codeexamplewidth%
        \else%
          \hsize\codeexamplewidth%
          \advance\hsize by2cm%
        \fi%
        \leavevmode\catcode`\^^M=9%
        \foreach \pgfmanualtime/\pgfmanualtimehow in\code@animation@list{%
          \setbox\codeexampleboxanim=\hbox{\colorbox{animationgraphicbackground}{%
              \tikzset{make snapshot of=\pgfmanualtime}%
              \scalebox{\pgfmanualanimscale}{\color{black}\ignorespaces%
                \code@animation@pre\expandafter\scantokens\expandafter{\code@temp\ignorespaces}\code@animation@post\ignorespaces}%
            }}%
          \space\raise4pt\hbox to0pt{\vrule width0pt height1em\hbox
            to\wd\codeexampleboxanim{\hfil\scriptsize$t{=}\pgfmanualtimehow \mathrm s$\hfil}\hss}%
          \lower\ht\codeexampleboxanim\box\codeexampleboxanim\hfil\penalty0\hskip0ptplus-1fil%
        }%
      }%
      \setbox\codeexampleboxanim=\hbox{\hbox{}\hskip-2cm\box\codeexampleboxanim}%
      \fi%
      \ifdim\wd\codeexamplebox>\codeexamplewidth%
        \def\code@start{\par}%
        \def\code@flushstart{}\def\code@flushend{}%
        \def\code@mid{\parskip2pt\par\noindent}%
        \def\code@width{\linewidth-6pt}%
        \def\code@end{}%
      \else%
        \def\code@start{%
%          \linewidth=\textwidth% removed this line
          \parshape \@ne 0pt \linewidth
          \leavevmode%
          \hbox\bgroup}%
        \def\code@flushstart{\hfill}%
        \def\code@flushend{\hbox{}}%
        \def\code@mid{\hskip6pt}%
        \def\code@width{\linewidth-12pt-\codeexamplewidth}%
        \def\code@end{\egroup}%
      \fi%
      \code@start%
      \noindent%
      \begin{minipage}[t]{\codeexamplewidth}\raggedright
        \hrule width0pt%
        \footnotesize\vskip-1em%
        \code@flushstart\box\codeexamplebox\code@flushend%
        \vskip0pt%
        \leavevmode%
        \box\codeexampleboxanim%
        \vskip-1ex
        \leavevmode%
      \end{minipage}%
    \else%
      \def\code@mid{\par}
      \def\code@width{\linewidth-6pt}
      \def\code@end{}
    \fi%
    \code@mid%
      \ifpgfmanual@multipage@code%
        {%
          \pgfkeysvalueof{/codeexample/prettyprint/base color}%
          \pgfmanualdolisting{#1}%
        }%
      \else%
        \colorbox{codebackground}{%
          \pgfkeysvalueof{/codeexample/prettyprint/base color}%
          \begin{minipage}[t]{\code@width}%
            \pgfmanualdolisting{#1}%
          \end{minipage}}%
      \fi%
    \code@end%
    \par%
    \medskip
  \fi
  \endcodeexample\endgroup%
}

\newenvironment{pathoperation'}[3][]{
  \begin{pgfmanualentry}
    \def\pgfmanualtest{#1}%
    \pgfmanualentryheadline{%
      \ifx\pgfmanualtest\@empty%
        \pgfmanualpdflabel{#2}{}%
      \fi%
      %\textcolor{gray}{{\ttfamily\char`\\path}\ \ \dots}
      %\declare{\texttt{\noligs{#2}}}#3\ \textcolor{gray}{\dots\texttt{;}}
    }%
    \ifx\pgfmanualtest\@empty%
      \index{#2@\protect\texttt{#2} path operation}%
      \index{Path operations!#2@\protect\texttt{#2}}%
    \fi%
    \pgfmanualbody
}
{
  \end{pgfmanualentry}
}
\makeatother

\usepackage{paracol}

%% normal references
\newcommand*{\indexLibrary}[1]{%
  \index{#1@\protect\texttt{#1} library}%
  \index{Libraries!#1@\protect\texttt{#1}}%
}
\newcommand*{\indexMathFunction}[1]{%
  \index{#1@\protect\texttt{#1} math function}%
  \index{Math functions!#1@\protect\texttt{#1}}%
}
\newcommand*{\indexShape}[1]{%
  \index{#1@\protect\texttt{#1} shape}%
  \index{Shapes!#1@\protect\texttt{#1}}%
}
\newcommand*{\indexHandler}[1]{%
  \index{\gobble#1@\protect\texttt{#1} handler}%
  \index{Key handlers!#1@\protect\texttt{#1}}
}
\newcommand*{\indexKey}[2][/tikz/]{%
\begingroup
  \def\mykey{#1#2}%
  \def\mypath{}%
  \gdef\myname{}%
  \firsttimetrue
  \decompose#1#2/\nil
\endgroup
}
\newcommand*{\indexPathOperation}[1]{%
  \index{#1@\protect\texttt{#1} path operation}%
  \index{Path operations!#1@\protect\texttt{#1}}%
}

\newcommand*{\indexPackage}[1]{%
  \index{#1@\protect\texttt{#1} package}%
  \index{Packages and files!#1@\protect\texttt{#1}}%
}
\newcommand*{\indexPackageO}[1]{%
  \index{#1@\originalprefix\protect\texttt{\original{#1}} package\originalsuffix}%
  \index{Packages and files!#1@\originalprefix\protect\texttt{\original{#1}}\originalsuffix}%
}

%% references to original
\newcommand*{\indexLibraryO}[1]{%
  \index{#1@\originalprefix\protect\texttt{\original{#1}} library\originalsuffix}%
  \index{Libraries!#1@\originalprefix\protect\texttt{\original{#1}}\originalsuffix}%
}
\newcommand*{\indexMathFunctionO}[1]{%
  \index{#1@\protect\texttt{\original{#1}} math function\originalsuffix}%
  \index{Math functions!#1@\protect\texttt{\original{#1}}\originalsuffix}%
}
\newcommand*{\indexShapeO}[1]{%
  \index{#1@\originalprefix\protect\texttt{\original{#1}} shape\originalsuffix}%
  \index{Shapes!#1@\originalprefix\protect\texttt{\original{#1}}\originalsuffix}%
}
\newcommand*{\indexHandlerO}[1]{%
  \index{\gobble#1@\originalprefix\protect\texttt{\original{#1}} handler\originalsuffix}%
  \index{Key handlers!#1@\originalprefix\protect\texttt{\original{#1}}\originalsuffix}
}
\newcommand*{\indexKeyO}[2][/tikz/]{%TODO
  \begingroup\let\decomposetoodeep\decomposetoodeepO
    \def\mykey{#1#2}%
    \def\mypath{}%
    \gdef\myname{}%
    \firsttimetrue
    \decomposeO#1#2/\nil
  \endgroup
}
\def\decomposetoodeepO#1/#2/\nil{%
  % avoid too-deep nesting in index
  \index{#1/#2@\originalprefix\protect\texttt{\original{#1/#2}} key\originalsuffix}%
  \index{\mypath#1/#2@\originalprefix\protect\texttt{\original{#1/#2}}\originalsuffix}%
  \decomposefindlast/#1/#2/\nil%
}
\def\decomposeO/#1/#2\nil{%
  \def\test{#2}%
  \ifx\test\empty%
    % aha.
    \index{#1@\originalprefix\protect\texttt{\original{#1}} key\originalsuffix}%
    \index{\mypath#1@\originalprefix\protect\texttt{\original{#1}}\originalsuffix}%
    \gdef\myname{#1}%
    \pgfmanualpdflabel{#1}{}
  \else%
    \advance\pgfmanualdecomposecount by1\relax%
    \ifnum\pgfmanualdecomposecount>2\relax%
      \decomposetoodeepO#1/#2\nil%
    \else%
      \iffirsttime%
        \begingroup%  
          % also make a pdf link anchor with full key path.
          \def\hyperlabelwithoutslash##1/\nil{%
            \pgfmanualpdflabel{##1}{}%
          }%
          \hyperlabelwithoutslash/#1/#2\nil%
        \endgroup%
        \def\mypath{#1@\protect\texttt{/#1/}!}%
        \firsttimefalse%
      \else%
        \expandafter\def\expandafter\mypath\expandafter{\mypath#1@\protect\texttt{#1/}!}%
      \fi%
      \def\firsttime{}%
      \decomposeO/#2\nil%
    \fi%
  \fi%
}
\newcommand*{\indexPathOperationO}[1]{%
  \index{#1@\originalprefix\protect\texttt{\original{#1}} path operation\originalsuffix}%
  \index{Path operations!#1@\originalprefix\protect\texttt{\original{#1}}\originalsuffix}%
}

\newcommand*{\indexCommand}[1]{%
  \removeats{#1}%
  \index{\strippedat @\protect\texttt{\protect\string\protect#1}}%
}
\newcommand*{\indexCommandO}[1]{%
  \removeats{#1}%
  \index{\strippedat @\originalprefix\protect\texttt{\original{\protect\string\protect#1}}\originalsuffix}%
}

\newcommand*{\indexDecorationO}[1]{%
  \index{#1@\originalprefix\protect\texttt{\original{#1}} decoration\originalsuffix}%
  \index{Decorations!#1@\originalprefix\protect\texttt{\original{#1}}\originalsuffix}%
}

\newcommand*{\indexEnvironment}[1]{%
  \index{#1@\protect\texttt{#1} environment}%
  \index{Environments!#1@\protect\texttt{#1}}%
}
\newcommand*{\indexEnvironmentO}[1]{%
  \index{#1@\originalprefix\protect\texttt{\original{#1}} environment\originalsuffix}%
  \index{Environments!#1@\originalprefix\protect\texttt{\original{#1}}\originalsuffix}%
}

\newcommand*{\indexArrowtipO}[1]{%
  \index{#1@\originalprefix\protect\texttt{\original{#1}} arrow tip\originalsuffix}%
  \index{Arrow tips!#1@\originalprefix\protect\texttt{\original{#1}}\originalsuffix}%
}
\newcommand*{\referenceArrowtipandIndexO}[1]{\texttt{#1}\indexArrowtipO{#1}}
\newcommand*{\referenceEnvironmentandIndexO}[1]{\texttt{#1}\indexEnvironmentO{#1}}
\newcommand*{\referenceDecorationandIndexO}[1]{\texttt{#1}\indexDecorationO{#1}}
\newcommand*{\referenceKeyandIndexO}[2][/tikz/]{\texttt{#1#2}\indexKeyO[#1]{#2}}
\newcommand*{\referenceKeyandIndex}[2][/tikz/]{\texttt{#1#2}\indexKey[#1]{#2}}
\newcommand*{\referenceShapeandIndex}[1]{\texttt{#1}\indexShape{#1}}
\newcommand*{\referenceShapeandIndexO}[1]{\texttt{#1}\indexShapeO{#1}}
\newcommand*{\referenceLibraryandIndexO}[1]{\texttt{#1}\indexLibraryO{#1}}
\newcommand*{\referenceLibraryandIndexExt}[1]{\texttt{#1}\indexLibraryExt{#1}}
\newcommand*{\referenceLibraryandIndex}[1]{\texttt{#1}\indexLibrary{#1}}
\newcommand*{\referencePackageandIndex}[1]{\texttt{#1}\indexPackage{#1}}
\newcommand*{\referencePackageandIndexO}[1]{\texttt{#1}\indexPackageO{#1}}
\newcommand*{\referencePackageandIndeXExt}[1]{\texttt{#1}\indexPackageExt{#1}}
\newcommand*{\referencePathOperationandIndex}[1]{\texttt{#1}\indexPathOperation{#1}}
\newcommand*{\referencePathOperationandIndexO}[1]{\texttt{#1}\indexPathOperationO{#1}}
\newcommand*{\referenceCommandandIndex}[1]{\texttt{\string#1}\indexCommand{#1}}
\newcommand*{\referenceCommandandIndexO}[1]{\texttt{\string#1}\indexCommandO{#1}}
\colorlet{originalindex}{gray}
\newcommand*{\original}[1]{\textcolor{originalindex}{#1}}%
\newcommand*{\originalsuffix}{}%\textsuperscript{\textdagger}}%\normalsize
\newcommand*{\originalprefix}{}%\footnotesize

\newcommand*{\indexPackageExt}[1]{%
  \begingroup
%    \originalExternals
    \index{#1@\originalExternals{\originalprefix\protect\texttt{\original{#1}} package\originalsuffix}}%
    \index{Packages and files!#1@\originalExternals{\originalprefix\protect\texttt{\original{#1}}\originalsuffix}}%
  \endgroup
}
\newcommand*{\indexLibraryExt}[1]{%
  \index{#1@\originalExternals{\originalprefix\protect\texttt{\original{#1}} library\originalsuffix}}%
  \index{Libraries!#1@\originalExternals{\originalprefix\protect\texttt{\original{#1}}\originalsuffix}}%
}
\DeclareRobustCommand*{\originalExternals}[1]{%
  \begingroup
    \colorlet{originalindex}{blue!50!gray}%
%    \renewcommand*{\original}[1]{\textcolor{originalindex}{##1}}%
    #1%
  \endgroup
}

\newcommand*{\filetype}[1]{\textsc{#1}}

\makeatletter
\def\index@prologue{\section*{Index}\addcontentsline{toc}{section}{Index}
  This index contains automatically generated entries as well as {\originalprefix\original{references}\originalsuffix} to
  original functionalities of \pgfname/\tikzname\space
  and \originalExternals{\originalprefix\original{references}\originalsuffix} to functionalities outside of \pgfname/\tikzname.
  \bigskip
}
\makeatother

\DeclareMathOperator{\sgn}{sgn}

\newenvironment{implementation}{%
  \noindent\rule[.5ex]{\linewidth}{.4pt}\par
  \paragraph*{Notes on Implementation:}\par%
  \small
  \begin{quotation}%
}{\end{quotation}}

\newcommand*{\inspiration}[2]{%
  \par\textbf{Q\,\&\,A:} \cite{#1} \& \cite{#2}\par
}
\newcommand*\inspirationQ[1]{\par\textbf{Q:} \cite{#1}}

\newenvironment{texpackage}[1]{% no context
  \begin{pgfmanualentry}
    \pgfmanualentryheadline{%
      \pgfmanualpdflabel{#1}{}%
      {\ttfamily\char`\\usepackage\char`\{\declare{#1}\char`\}\space\space \char`\%\space\space  \LaTeX}}
    \index{#1@\protect\texttt{#1} package}%
    \index{Packages and files!#1@\protect\texttt{#1}}%
    \pgfmanualentryheadline{{\ttfamily\char`\\input \declare{#1}.tex\space\space\space \char`\%\space\space  plain \TeX}}
%    \pgfmanualentryheadline{{\ttfamily\char`\\usemodule[\declare{#1}]\space\space \char`\%\space\space  Con\TeX t}}
    \pgfmanualbody
}
{
  \end{pgfmanualentry}
}
\newenvironment{latexpackage}[1]{
  \begin{pgfmanualentry}
    \pgfmanualentryheadline{%
      \pgfmanualpdflabel{#1}{}%
      {\ttfamily\char`\\usepackage\char`\{\declare{#1}\char`\}\space\space \char`\%\space\space  \LaTeX}}
    \index{#1@\protect\texttt{#1} package}%
    \index{Packages and files!#1@\protect\texttt{#1}}%
%    \pgfmanualentryheadline{{\ttfamily\char`\\input \declare{#1}.tex\space\space\space \char`\%\space\space  plain \TeX}}
%    \pgfmanualentryheadline{{\ttfamily\char`\\usemodule[\declare{#1}]\space\space \char`\%\space\space  Con\TeX t}}
    \pgfmanualbody
}
{
  \end{pgfmanualentry}
}

\usepackage{booktabs}

\renewenvironment{arrowtip}[4]{
  \begin{pgfmanualentry}
    \pgfmanualentryheadline{\textbf{Arrow Tip Kind} {\ttfamily#1}}
    \index{#1@\protect\texttt{#1} arrow tip}%
    \index{Arrow tips!#1@\protect\texttt{#1}}%
    \pgfmanualbody
    \def\currentarrowtype{#1}
    \begin{minipage}[t]{\dimexpr\linewidth-5cm\relax}
      #2
    \end{minipage}\hskip5mm\begin{minipage}[t]{4.75cm}
      \leavevmode\vskip-2em
    \tikz{
      \draw [black!50,line width=5mm,-{#1[#3,color=black]}] (-4,0) -- (0,0);
      \foreach \action in {#4}
      { \expandafter\processaction\action\relax }
    }
    \end{minipage}\par\smallskip
  }
{
  \end{pgfmanualentry}
}
\makeatletter
\def\arrowexample@#1[#2]{%
  {\sfcode`\.1000\scriptsize\texttt{#1[#2]}} \>
  \tikz [baseline,>={#1[#2]}] \draw [line
  width=0.4pt,->] (0,.5ex) -- (2em,.5ex); thin \>
  \tikz [baseline,>={#1[#2]}] \draw [line
  width=0.8pt,->] (0,.5ex) -- (2em,.5ex); \textbf{thick} \>
  \tikz [baseline,>={#1[#2]}] \draw [line
  width=1.6pt,->] (0,.5ex) -- (3em,.5ex); \\
}
%\def\arrowexampledup[#1]{\arrowexample[{#1] \currentarrowtype[}]}
%\def\arrowexampledupdot[#1]{\arrowexample[{#1] . \currentarrowtype[}]}
\makeatother