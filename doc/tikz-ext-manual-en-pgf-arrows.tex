% !TeX spellcheck = en_US
% !TeX root = tikz-ext-manual.tex
% Copyright 2023 by Qrrbrbirlbel
%
% This file may be distributed and/or modified
%
% 1. under the LaTeX Project Public License and/or
% 2. under the GNU Free Documentation License.
%

\section{Untipped Arrow Tips}
\label{pgflibrary:arrows}
\tikzset{external/export/.try=false}%
\begin{pgflibrary}{ext.arrows}
  This library adds arrows to \pgfname/\tikzname.
\end{pgflibrary}

The arrow tips of the \referenceLibraryandIndexO{arrows.meta} library always
just touch the end of original line -- which is usually
what you want.

But for some arrow tips (and when they lie along a path) it makes sense
that these tips shoot a bit over the end of the line.
This is why these arrow tips exist.
They can be categorized into three groups:
\begin{enumerate}
\item Centered
\item Untipped
\item Overtipped\footnote{The Overtipped arrow tips aren't yet implemented.}
\end{enumerate}
Not all original arrow tips got all variants.
For a summary, refer to table~\ref{tab:tips}.
\begin{table}
\newcommand*\tip[2][]{%
  \tikzset{external/export/.try=false}%
  \tikz[arrows={#2 \arrowtiprow[reversed] - \arrowtiprow[sep] . #2 \arrowtiprow[]}, baseline=+-.5ex,
    trim left=+-3mm, trim right=+23mm]
    \draw[line width=+.5mm, postaction={draw, gray, thin,-}] (0,0) -- + (right:2);%
}%
\def\tiprow#1 &{#1\gdef\arrowtiprow{#1} &}%
\def\Tiprow#1 &{\quad#1\gdef\arrowtiprow{#1} &}%
\centering
\caption{Variants of the original \texttt{arrows.meta} arrow tips.
  For each example, the order is 1. reversed variant, 2. original, 3. variant.
  They gray line shows where the line actually ends.
}\label{tab:tips}
\begin{tabular}{ll ccc}
  \toprule
  Group        & Original                   & Centered       &    Untipped    & Overtipped \\ \midrule
  Barbed       & \tiprow Arc Barb           & \tip{Centered} & \tip{Untipped} &     --     \\
               & \Tiprow Parenthesis        & \tip{Centered} & \tip{Untipped} &     --     \\
               & \tiprow Hooks              & \tip{Centered} &       --       &     --     \\
               & \tiprow Straight Barb      & \tip{Centered} &       --       &     --     \\
               & \tiprow Tee Barb           & \tip{Centered} & \tip{Untipped} &     --     \\
               & \Tiprow Bar                & \tip{Centered} & \tip{Untipped} &     --     \\
               & \Tiprow Bracket            & \tip{Centered} & \tip{Untipped} &     --     \\ \midrule
  Mathematical & Classical TikZ Rightarrow  & --             &       --       &     --     \\
               & Computer Modern Rightarrow & --             &       --       &     --     \\
               & \quad To                   & --             &       --       &     --     \\
               & Implies                    & --             &       --       &     --     \\ \midrule
  Geometric    & \tiprow Circle             & \tip{Centered} & \tip{Untipped} &     --     \\
               & \Tiprow Ellipse            & \tip{Centered} & \tip{Untipped} &     --     \\
               & \tiprow Kite               & \tip{Centered} &       --       &     --     \\
               & \Tiprow Diamond            & \tip{Centered} &       --       &     --     \\
               & \Tiprow Turned Square      & \tip{Centered} &       --       &     --     \\
               & \tiprow LaTeX              & --             &       --       &     --     \\
               & \tiprow Square             & \tip{Centered} &       --       &     --     \\
               & \Tiprow Rectangle          & \tip{Centered} &       --       &     --     \\
               & \tiprow Stealth            & \tip{Centered} &       --       &     --     \\
               & \Tiprow Triangle           & \tip{Centered} &       --       &     --     \\ \midrule
  Caps         &                            & --             &       --       &     --     \\
  Rays         & \tiprow Rays               & \tip{Centered} &       --       &     --     \\ \bottomrule
\end{tabular}
\end{table}
\begin{multicols}{2}
As with the original tips of the \referenceLibraryandIndexO{arrows.meta} library
these can be organized in the following categories.
\subsection{Centered}
\subsubsection{Barbed Arrow Tips}
\begin{arrowtipsimple}{Centered Arc Barb}
    This is a variant of the \referenceArrowtipandIndexO{Arc Barb} tip.
    The center of the arc lies on the original end of the path.

    \begin{arrowexamples}
        \arrowexample[]
        \arrowexampledup[sep]
        \arrowexampledupdot[sep]
        \arrowexample[arc=120]
        \arrowexample[arc=270]
        \arrowexample[length=2pt]
        \arrowexample[length=2pt,width=5pt]
        \arrowexample[line width=2pt]
        \arrowexample[reversed]
        \arrowexample[round]
        \arrowexample[slant=.3]
        \arrowexample[left]
        \arrowexample[right]
        \arrowexample[harpoon,reversed]
        \arrowexample[red]
    \end{arrowexamples}
    %
    The following options have no effect: |open|, |fill|.

    On |double| lines, the arrow tip will not look correct.
\end{arrowtipsimple}

\begin{arrowtipsimple}{Centered Bar}
    A simple bar. This is a simple instance of |Centered Tee Barb| for length zero.
    
    The middle of the line will lie on original end of the path.
\end{arrowtipsimple}

\begin{arrowtip}{Centered Bracket}{
    This is an instance of the |Centered Tee Barb| arrow tip that results in something
    resembling a bracket.
    
    The middle of the vertical part will lie on the original end of the path.
}%
{}%
{}

    \begin{arrowexamples}
        \arrowexample[]
        \arrowexampledup[sep]
        \arrowexampledupdot[sep]
        \arrowexample[reversed]
        \arrowexample[round]
        \arrowexample[slant=.3]
        \arrowexample[left]
        \arrowexample[right]
        \arrowexample[harpoon,reversed]
        \arrowexample[red]
    \end{arrowexamples}
    %
    The following options have no effect: |open|, |fill|.

    On |double| lines, the arrow tip will not look correct.
\end{arrowtip}
\end{multicols}
\tikzset{external/export/.try=true}%
\endinput
\begin{itemize}
\item \emph{Barbed} arrow tips
  \begin{arrowexamples}
    \arrowexample Centered Arc Barb[]
    \arrowexample Centered Bar[]
    \arrowexample Centered Bracket[]
    \arrowexample Centered Hooks[]
    \arrowexample Centered Parenthesis[]
    \arrowexample Centered Straight Barb[]
    \arrowexample Centered Tee Barb[]
  \end{arrowexamples}
\item \emph{Mathematical} arrow tips don't have any variants.
\item \emph{Geometric} arrow tips.
  The |Stealth| variant is \enquote{centered} on its centroid.
  The |Kite| is \enquote{centered} around its widest part.
  \begin{arrowexamples}
      \arrowexample Centered Circle[]
      \arrowexample Centered Diamond[]
      \arrowexample Centered Ellipse[]
      \arrowexample Centered Kite[]
      \arrowexample Centered Rectangle[]
      \arrowexample Centered Square[]
      \arrowexample Centered Stealth[]
      \arrowexample Centered Stealth[round]
      \arrowexample Centered Triangle[]
      \arrowexample Centered Turned Square[]
  \end{arrowexamples}
  \begin{arrowexamples}
      \arrowexample Centered Circle[open]
      \arrowexample Centered Diamond[open]
      \arrowexample Centered Ellipse[open]
      \arrowexample Centered Kite[open]
      \arrowexample Centered Rectangle[open]
      \arrowexample Centered Square[open]
      \arrowexample Centered Stealth[open]
      \arrowexample Centered Stealth[round]
      \arrowexample Centered Triangle[open]
      \arrowexample Centered Turned Square[open]
  \end{arrowexamples}
\item No \emph{Caps} tip have variants.
\item \emph{Rays} have the Centered variant.
  \begin{arrowexamples}
    \arrowexample Centered Rays[]
  \end{arrowexamples}
\end{itemize}