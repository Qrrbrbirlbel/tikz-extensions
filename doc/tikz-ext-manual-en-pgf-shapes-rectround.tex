% !TeX spellcheck = en_US
% !TeX root = tikz-ext-manual.tex
% Copyright 2022 by Qrrbrbirlbel
%
% This file may be distributed and/or modified
%
% 1. under the LaTeX Project Public License and/or
% 2. under the GNU Free Documentation License.
%

\section{Shape: Rectangle with Rounded Corners}
\begin{pgflibrary}{ext.shapes.rectangleroundedcorners}
  A rectangle with rounded corners.
\end{pgflibrary}

\begin{key}{/pgf/rectangle with rounded corners north west radius=\meta{dimen} (initially .5\string\pgflinewidth)}
  Sets the north west radius to \meta{dimen}.
\end{key}
\begin{key}{/pgf/rectangle with rounded corners north east radius=\meta{dimen} (initially .5\string\pgflinewidth)}
  Sets the north east radius to \meta{dimen}.
\end{key}
\begin{key}{/pgf/rectangle with rounded corners south west radius=\meta{dimen} (initially .5\string\pgflinewidth)}
  Sets the south west radius to \meta{dimen}.
\end{key}
\begin{key}{/pgf/rectangle with rounded corners south east radius=\meta{dimen} (initially .5\string\pgflinewidth)}
  Sets the south east radius to \meta{dimen}.
\end{key}
\begin{key}{/pgf/rectangle with rounded corners radius=\meta{dimen}}
  Sets all radii to \meta{dimen}.
\end{key}

\begin{codeexample}[preamble=\usepgflibrary{ext.shapes.rectangleroundedcorners}]
\begin{tikzpicture}\Huge
\node[name=s, shape=rectangle with rounded corners, shape example,
  rectangle with rounded corners north west radius=10pt,
  rectangle with rounded corners north east radius=20pt,
  rectangle with rounded corners south west radius=30pt,
  rectangle with rounded corners south east radius=40pt] {Rectangle with rounded corners\vrule width 1pt height 2cm};
\foreach \anchor/\placement in
  {north west/above left, north/above, north east/above right,
         west/left,      center/above,       east/right,
     mid west/right,        mid/above,   mid east/left,
    base west/left,        base/below,  base east/right,
   south west/below left, south/below, south east/below right,
   text/below, 10/right, 130/above,
   north west center/below right,      north east center/left,
   south west center/above right,      south east center/left,
   below north west/left,  above south west/left, above south east/right, below north east/right,
   right north west/above, right south west/below, left south east/below,  left north east/above}
   \draw[shift=(s.\anchor)] plot[mark=x] coordinates{(0,0)}
     node[\placement] {\scriptsize\texttt{(s.\anchor)}};
\end{tikzpicture}
\end{codeexample}