% Copyright 2023 by Qrrbrbirlbel
%
% This file may be distributed and/or modified
%
% 1. under the LaTeX Project Public License and/or
% 2. under the GNU Free Documentation License.
%

% \tikzext@arrow@evalshift
%
% Stores the shift amount into \tikzext@arrow@shift@
%
\def\tikzext@arrow@evalshift#1{%
  \ifcase\pgfkeysvalueof{/tikz-ext/arrow/shift mode}\relax
    \def\tikzext@arrow@shift@{0pt}%
  \or
    \pgfarrowtotallength{#1}%
    \pgfmathsetlengthmacro\tikzext@arrow@shift@{(\pgfkeysvalueof{/tikz/arrow shift factor})*-\pgf@x}%
  \or
    \pgfarrowtotallength{#1}%
    \pgfmathsetlengthmacro\tikzext@arrow@shift@{(\pgfkeysvalueof{/tikz/arrow shift factor})*-\pgf@xa}%
  \fi
}

% This redefinition is necessary to save the center of an arc
% for the arc arrow
\let\tikzext@arrow@pgfpointarcaxesattime\pgfpointarcaxesattime
\def\pgfpointarcaxesattime#1#2#3#4#5#6{%
  \pgfgettransform\tikzext@arrow@transform
  \pgfextract@process\tikzext@arrow@center{#2}%
  \tikzext@arrow@pgfpointarcaxesattime{#1}{#2}{#3}{#4}{#5}{#6}%
}

%%
% For bended arrows the shifting can't simply be done by a transformation
% it must be done by the arrow system itself via the sep key.
% For this to work as expected, the sep will need to be sneaked into
% the options of the first or last arrow tip.
\def\tikzext@arrow@doshift#1-#2\pgf@stop{%
  \pgfutil@ifempty{#1}{\pgfsetarrowsstart{}}{%
    \ifcase\pgfkeysvalueof{/tikz-ext/arrow/shift mode}\relax
      \pgfsetarrowsstart{#1}%
    \else
      \tikzext@arrow@evalshift{#1}%
      \let\tikz@temp\pgfutil@empty
      \tikzext@arrow@sneakstart{#1}{sep=+\tikzext@arrow@shift@}%
      \pgfsetarrowsstart{\tikz@temp}%
    \fi
  }%
  \pgfutil@ifempty{#2}{\pgfsetarrowsend{}}{%
    \ifcase\pgfkeysvalueof{/tikz-ext/arrow/shift mode}\relax
      \pgfsetarrowsend{#2}%
    \else
      \tikzext@arrow@evalshift{#2}%
      \let\tikz@temp\pgfutil@empty
      \tikzext@arrow@sneakend{#2}{sep=+\tikzext@arrow@shift@}%
      \pgfsetarrowsend{\tikz@temp}%
    \fi
  }%
}
% \tikzext@arrow@sneakend{<tip>}{<sneak>}
%
% Sneaks <sneak> at the end of the <tip> specification in
% so that it only applies to the last tip
%
\def\tikzext@arrow@sneakend#1#2{%
  \pgfutil@in@{]}{#1}%
  \ifpgfutil@in@
    \expandafter\pgfutil@firstoftwo
  \else
    \expandafter\pgfutil@secondoftwo
  \fi{%
    \tikzext@arrow@sneakend@#1\pgf@stop{#2}%
  }{%
    \expandafter\def\expandafter\tikz@temp\expandafter{\tikz@temp#1[#2]}%
  }%
}
\def\tikzext@arrow@sneakend@#1]#2\pgf@stop#3{%
  \pgfutil@ifempty{#2}{%
    \expandafter\def\expandafter\tikz@temp\expandafter{\tikz@temp#1,#3]}%
  }{%
    \expandafter\def\expandafter\tikz@temp\expandafter{\tikz@temp#1]}%
    \tikzext@arrow@sneakend{#2}{#3}%
  }%
}
% \tikzext@arrow@sneakstart{<tip>}{<sneak>}
%
% Sneaks <sneak> at the start of the <tip> specification in
% so that it only applies to the first tip
%
\def\tikzext@arrow@sneakstart#1#2{%
  \pgfutil@in@{]}{#1}%
  \ifpgfutil@in@
    \tikzext@arrow@sneakstart@#1\pgf@stop{#2}%
  \else
    \def\tikz@temp{#1[#2]}%
  \fi
}
\def\tikzext@arrow@sneakstart@#1]#2\pgf@stop#3{%
  \def\tikz@temp{#1,#3]#2}%
}

\let\tikzext@processarrows@orig\tikz@processarrows
\def\tikzext@processarrows#1{%
  \def\tikz@current@arrows{#1}%
  \pgfutil@ifempty{#1}{}{%
    \pgfutil@in@{-}{#1}%
    \ifpgfutil@in@
      \def\tikzext@current@arrows{#1}%
    \else
      \pgfsetarrows{#1}%
    \fi
  }%
}

%
% Arrow keys
%
\tikzset{%
  softpath arrows/.default={/tikz/arrows=->},
  softpath arrows/.code={%
    \begingroup
      \let\tikz@processarrows\tikzext@processarrows
      \tikzset{#1}%
      \pgfsyssoftpath@getcurrentpath\tikz@temp
      \pgfprocesssplitsubpath\tikz@temp
      \expandafter\pgf@parse@end\pgfprocessresultsubpathsuffix\pgf@stop\pgf@stop\pgf@stop
      \tikzext@arrow@softpath@curve
      \setbox\tikz@whichbox=\hbox\bgroup
        \unhbox\tikz@whichbox
        \hbox\bgroup\bgroup
          \pgfinterruptpath
          \pgfscope
            \expandafter\tikzext@arrow@doshift\tikzext@current@arrows\pgf@stop
            \pgfset{tips}%
            \pgftransformreset
            \tikzext@arrow@softpath
            \pgfusepath{}%
          \endpgfscope
          \endpgfinterruptpath
        \egroup\egroup
      \egroup
    \global\setbox\tikz@tempbox=\box\tikz@whichbox%
    \expandafter\endgroup%
    \expandafter\setbox\tikz@whichbox=\box\tikz@tempbox%
  },
  pos </.initial=0.0,
  pos >/.style={/tikz/pos={#1}},
  arrow shift mode/.is choice,
  arrow shift mode/off/.code                  =\pgfkeyssetvalue{/tikz-ext/arrow/shift mode}{0},
  arrow shift mode/total length/.code         =\pgfkeyssetvalue{/tikz-ext/arrow/shift mode}{1},
  arrow shift mode/total/.code                =\pgfkeyssetvalue{/tikz-ext/arrow/shift mode}{1},
  arrow shift mode/length until line end/.code=\pgfkeyssetvalue{/tikz-ext/arrow/shift mode}{2},
  arrow shift mode/line end/.code             =\pgfkeyssetvalue{/tikz-ext/arrow/shift mode}{2},
  arrow shift mode=total length,
  arrow shift factor/.initial=.5,
%
% Arrow pics
%
  pics/arrow/.default=>,
  pics/arrow/.style={
    /tikz/sloped, /tikz/allow upside down,
    foreground code=, background code=, setup code=,
    code=%
      \tikzext@arrow@evalshift{#1}%
      \pgftransformxshift{+-\tikzext@arrow@shift@}%
      \pgfarrowdraw{#1}%
  },
  pics/arc arrow/.default=>,
  pics/arc arrow/.style={/tikz/pics/arc arrows=-{#1}},
  pics/arc arrows/.default=->,
  pics/arc arrows/.style={
    foreground code=, background code=,
    setup code={\tikzext@arrow@doshift#1\pgf@stop},
    code=%
      \pgfset{tips}%
      \pgftransformreset
      \pgfsettransform\tikzext@arrow@transform
      \pgfpathmoveto{%
        \tikzext@arrow@pgfpointarcaxesattime
          {\pgfkeysvalueof{/tikz/pos <}}
          {\tikzext@arrow@center}
          {\tikz@timer@zero@axis}
          {\tikz@timer@ninety@axis}
          {\tikz@timer@start@angle}
          {\tikz@timer@end@angle}%
      }%
      \pgfpatharcaxes
        {(\tikz@timer@end@angle-\tikz@timer@start@angle)*%
             (\pgfkeysvalueof{/tikz/pos <})+\tikz@timer@start@angle}
        {(\tikz@timer@end@angle-\tikz@timer@start@angle)*%
             (\tikz@time)+\tikz@timer@start@angle}
        {\tikz@timer@zero@axis}
        {\tikz@timer@ninety@axis}%
      \pgfusepath{}%
  },
  pics/softpath arrow/.default=>,
  pics/softpath arrow/.style={/tikz/pics/softpath arrows=-{#1}},
  pics/softpath arrows/.default=->,
  pics/softpath arrows/.style={
    foreground code=, background code=,
    setup code={%
      \tikzext@arrow@doshift#1\pgf@stop
      \pgfsyssoftpath@getcurrentpath\tikz@temp
      \pgfprocesssplitsubpath\tikz@temp
      \expandafter\pgf@parse@end\pgfprocessresultsubpathsuffix\pgf@stop\pgf@stop\pgf@stop
      \tikzext@arrow@softpath@curve
    },
    code=%
      \pgfset{tips}%
      \pgftransformreset
      \tikzext@arrow@softpath
      \pgfusepath{}%
  }
}

% curve
\def\tikzext@arrow@softpath@curve{%
  \ifx\pgfpointthirdlastonpath\relax
    \let\tikzext@arrow@softpath\tikzext@arrow@linepath
  \else
    \let\tikzext@arrow@softpath\tikzext@arrow@curvepath
  \fi
}
% |- and -|
\def\tikzext@arrow@softpath@hvline{%
  \let\tikzext@lastonpath      \pgfpointlastonpath
  \let\tikzext@secondlastonpath\pgfpointsecondlastonpath
  \pgfprocesssplitpath\pgfprocessresultsubpathprefix
  \pgfprocesssplitsubpath\pgfprocessresultpathsuffix
  \expandafter\pgf@parse@end\pgfprocessresultsubpathsuffix\pgf@stop\pgf@stop\pgf@stop
  \let\tikzext@arrow@softpath\tikzext@arrow@hvpath
}
\let\tikzext@arrow@softpath@vhline\tikzext@arrow@softpath@hvline
%
% Paths
%
% Line
\def\tikzext@arrow@linepath{%
  \pgfpathmoveto{%
    \pgfpointlineattime
      {\pgfkeysvalueof{/tikz/pos <}}
      {\pgfpointsecondlastonpath}
      {\pgfpointlastonpath}%
  }%
  \pgfpathlineto{%
    \pgfpointlineattime
      {\tikz@time}
      {\pgfpointsecondlastonpath}
      {\pgfpointlastonpath}%
  }%
}
% Curve
\def\tikzext@arrow@curvepath{%
  \pgfpathcurvebetweentime
    {\pgfkeysvalueof{/tikz/pos <}}
    {\tikz@time}
    {\pgfpointfourthlastonpath}
    {\pgfpointthirdlastonpath}
    {\pgfpointsecondlastonpath}
    {\pgfpointlastonpath}%
}

%
% \arrow(reversed)*(*) macro for decoration.markings
%
\def\pgf@lib@dec@doarrowhead{%
  \pgfutil@ifnextchar*%
    {\tikzext@lib@dec@doarrowhead{1}}
    {\pgfutil@ifnextchar[%
      {\tikz@lib@dec@doarrowhead}
      {\tikz@lib@dec@doarrowhead[]}%
    }%
}
\def\pgf@lib@dec@doarrowheadrev{%
  \pgfutil@ifnextchar*%
    {\tikzext@lib@dec@doarrowhead{-1}}
    {\pgfutil@ifnextchar[%
      {\tikz@lib@dec@doarrowheadrev}
      {\tikz@lib@dec@doarrowheadrev[]}%
    }%
}

\def\tikzext@lib@dec@doarrowhead#1*{%
  \pgfutil@ifnextchar*%
    {\tikzext@lib@dec@doarrowhead@@{#1}}
    {\pgfutil@ifnextchar[%
      {\tikzext@lib@dec@doarrowhead@{#1}}
      {\tikzext@lib@dec@doarrowhead@{#1}[]}}%
}
\def\tikzext@lib@dec@doarrowhead@#1[#2]#3{%
  \scope[{#2}]%
    \tikzext@arrow@evalshift{#3}%
    \pgftransformxshift{+\ifnum#1>0 -\fi\tikzext@arrow@shift@}%
    \pgf@lib@dec@arrowhead{#1}{#3}%
  \endscope
}
\def\tikzext@lib@dec@doarrowhead@@#1*{%
  \pgfutil@ifnextchar[%]
    {\tikzext@lib@dec@doarrowhead@@@{#1}}
    {\tikzext@lib@dec@doarrowhead@@@{#1}[]}%
}

\def\tikzext@lib@dec@doarrowhead@@@#1[#2]#3{%
  \ifx\pgfdecorationcurrentinputsegment\pgfdecorationinputsegmentcurveto
    \pgftransformreset
    \pgfset{tips}%
    \pgfsetarrows{[bend]}%
    \ifnum#1<0
      \pgfsetarrows{[reversed]}%
    \fi
    \scope[{#2}]%
      \tikzext@arrow@doshift-#3\pgf@stop
      \ifdim\pgf@decorate@inputsegmenttime pt=1pt % something weird for 1.0
        \def\pgf@decorate@inputsegmenttime{1}%
      \fi
      \ifdim\pgf@decorate@inputsegmenttime pt<0.05pt % zero is bad
        \def\pgf@decorate@inputsegmenttime{0.05}%
      \fi
      \pgfpathcurvebetweentime{0}{\pgf@decorate@inputsegmenttime}%
          {\pgf@decorate@inputsegment@first}{\pgf@decorate@inputsegment@supporta}%
          {\pgf@decorate@inputsegment@supportb}{\pgf@decorate@inputsegment@last}%
      \pgfusepath{}%
    \endscope
  \else
    \tikzext@lib@dec@doarrowhead@{#1}[{#2}]{#3}%
  \fi
}