% Copyright 2024 by Qrrbrbirlbel
%
% This file may be distributed and/or modified
%
% 1. under the LaTeX Project Public License and/or
% 2. under the GNU Free Documentation License.
%

\unless\ifcsname tikzextset\endcsname
  \input tikzext-util.tex
\fi

\tikzextset{
  ignore line width/.value forbidden,
  ignore line width/.style={
    /pgf/local bounding box=tikzext@ilw,
    /tikz/execute at end scope/.expanded={%
      \global\pgf@picminx=\the\pgf@picminx\relax
      \global\pgf@picmaxx=\the\pgf@picmaxx\relax
      \global\pgf@picminy=\the\pgf@picminy\relax
      \global\pgf@picmaxy=\the\pgf@picmaxy\relax
    },
    /tikz/execute at end scope={%
      \pgf@protocolsizes\pgf@lbb@minx@tikzext@ilw\pgf@lbb@miny@tikzext@ilw
      \pgf@protocolsizes\pgf@lbb@maxx@tikzext@ilw\pgf@lbb@maxy@tikzext@ilw
    }%
  },
  max bounding box/.style={
    /tikz/execute at begin picture={%
      \pgfutil@IfUndefined{tikzext@maxbb@#1@next}{}{%
        \pgfkeys@expanded{\noexpand\tikzext@@maxbb@get\csname tikzext@maxbb@#1@next\endcsname}%
      }%
    },
    /tikz/execute at end picture={%
      \pgfutil@IfUndefined{tikzext@maxbb@#1@prev}{%
        \tikzext@atenddocument{%
          \immediate\write\pgfutil@auxout{%
            \noexpand\expandafter\gdef\noexpand\csname tikzext@maxbb@#1@next\endcsname
              {\csname tikzext@maxbb@#1@prev\endcsname}%
          }%
        }%
        \tikzext@@maxbb@set{#1}%
      }{%
        \expandafter\expandafter\expandafter\tikzext@@maxbb@get\csname tikzext@maxbb@#1@prev\endcsname
        \tikzext@@maxbb@set{#1}%
      }%
      \pgfutil@IfUndefined{tikzext@maxbb@#1@next}{}{%
        \expandafter\expandafter\expandafter\tikzext@@maxbb@get\csname tikzext@maxbb@#1@next\endcsname
      }%
    }%
  }%
}
\def\tikzext@@maxbb@get#1#2#3#4{%
  \ifdim#1<\pgf@picminx\global\pgf@picminx#1\fi
  \ifdim#3>\pgf@picmaxx\global\pgf@picmaxx#3\fi
  \ifdim#2<\pgf@picminy\global\pgf@picminy#2\fi
  \ifdim#4>\pgf@picmaxy\global\pgf@picmaxy#4\fi}
\def\tikzext@@maxbb@set#1{%
  \expandafter\xdef\csname tikzext@maxbb@#1@prev\endcsname
    {{\the\pgf@picminx}{\the\pgf@picminy}{\the\pgf@picmaxx}{\the\pgf@picmaxy}}%
}%

\ifcsname @ifclassloaded\endcsname % latex?
  \let\tikzext@atenddocument\AtEndDocument
  \@ifclassloaded{beamer}{% beamer?
  }{%
    \expandafter\endinput
  }%
\else
  \def\tikzext@atenddocument#1{%
    \pgfutil@everybye\expandafter{\the\pgfutil@everybye#1}%
  }%
  \expandafter\endinput
\fi

\newcounter{tikzextbb}
\resetcounteronoverlays{tikzextbb}
\tikzextset{
  sync bounding box/.code={%
    \stepcounter{tikzextbb}%
    \tikzextset{max bounding box={beamer-\the\value{tikzextbb}}}%
  }
}

\long\def\tikzext@beamerfunction@alt      #1#2{\alt      <#1>{\tikz@@command@path#2;}{\tikz@path@do@at@end}}
\long\def\tikzext@beamerfunction@uncover  #1#2{\uncover  <#1>{\tikz@@command@path#2;}\tikz@path@do@at@end}
\long\def\tikzext@beamerfunction@visible  #1#2{\visible  <#1>{\tikz@@command@path#2;}\tikz@path@do@at@end}
\long\def\tikzext@beamerfunction@invisible#1#2{\invisible<#1>{\tikz@@command@path#2;}\tikz@path@do@at@end}
\let\tikzext@beamerfunction\tikzext@beamerfunction@alt

% Overwriting original dispatchers
\long\def\tikz@eargnormalsemicolon<#1>#2;{\tikzext@beamerfunction{#1}{#2}}%
\begingroup
  \catcode`\;=\active
  \long\global\def\tikz@eargactivesemicolon<#1>#2;{\tikzext@beamerfunction{#1}{#2}}%
\endgroup

\tikzextset{
  /utils/ext/only/.code 2 args={\only<#1>{\pgfkeysalso{#2}}},
  /utils/ext/alt/.code args={#1#2#3}{\alt<#1>{\pgfkeysalso{#2}}{\pgfkeysalso{#3}}},
  /utils/ext/temporal/.code n args={4}{\temporal<#1>{\pgfkeysalso{#2}{\pgfkeysalso{#3}{\pgfkeysalso{#4}}}}},
  % beamer/temporal/.code args={#1#2#3#4}{\temporal<#1>{\pgfkeysalso{#3}}}
  beamer function/.is choice,
  beamer function/original/.code =\let\tikzext@beamerfunction\tikzext@beamerfunction@alt,
  beamer function/alt/.code      =\let\tikzext@beamerfunction\tikzext@beamerfunction@alt,
  beamer function/only/.code     =\let\tikzext@beamerfunction\tikzext@beamerfunction@alt,
  beamer function/uncover/.code  =\let\tikzext@beamerfunction\tikzext@beamerfunction@uncover,
  beamer function/visible/.code  =\let\tikzext@beamerfunction\tikzext@beamerfunction@visible,
  beamer function/invisible/.code=\let\tikzext@beamerfunction\tikzext@beamerfunction@invisible,
  uncover/.default=all,
  uncover/.code=%
    \tikz@addoption{%
      \beamer@noargsonslide<#1>%
      \aftergroup\beamer@noargsonslide
    },
  cover/.default=all,
  cover/.code=%
    \tikz@addoption{%
      \beamer@endpause
      \beamer@alt<#1>{\beamer@startcovered\gdef\beamer@endpause{\beamer@endcovered}}%
                     {\beamer@spacingcover\gdef\beamer@endpause{\beamer@spacingcover}}%
      \aftergroup\beamer@noargsonslide
    },
  aobs visible/.style={/utils/ext/alt={#1}{}{/tikz/opacity=0,/tikz/text opacity=0}},
  aobs invisible/.style={/utils/ext/only={#1}{/tikz/opacity=0,/tikz/text opacity=0}},
  visible/.default=all,
  visible/.code=%
    \tikz@addoption{%
      \beamer@noargsvisibleonslide<#1>%
      \aftergroup\beamer@noargsonslide
    },
  invisible/.default=all,
  invisible/.code=%
    \tikz@addoption{%
      \beamer@endpause
      \beamer@alt<#1>{\beamer@begininvisible\gdef\beamer@endpause{\beamer@endinvisible}}%
                     {\beamer@spacingcover\gdef\beamer@endpause{\beamer@spacingcover}}%
      \aftergroup\beamer@noargsonslide
    },
         uncover'/.style={/tikz/ext/cover={#1}},
    aobs visible'/.style={/tikz/ext/aobs invisible={#1}},
         visible'/.style={/tikz/ext/invisible={#1}},
           cover'/.style={/tikz/ext/uncover={#1}},
       invisible'/.style={/tikz/ext/visible={#1}},
  aobs invisible'/.style={/tikz/ext/aobs visible={#1}},
  /handlers/.ext_</.code=\tikzext@beamer@handler#1\pgf@stop,
  beamer shortcuts/.code=\pgfqkeys{/tikz/ext/beamer shortcuts}{#1},
  beamer shortcuts/aot/.code={%
    \pgfkeysdef{/tikz/alt}{\pgfkeysvalueof{/utils/ext/alt/.@cmd}##1\pgfeov}%
    \pgfkeysdef{/tikz/only}{\pgfkeysvalueof{/utils/ext/only/.@cmd}##1\pgfeov}%
    \pgfkeysdef{/tikz/temporal}{\pgfkeysvalueof{/utils/ext/temporal/.@cmd}##1\pgfeov}%
  },
  beamer shortcuts/first char/.is choice,
  beamer shortcuts/@enable first char/.code 2 args={%
    \pgfkeys@syntax@handlerstrue % \pgfkeys{/handlers/first char syntax=true}%
    \pgfkeyssetvalue{/handlers/first char syntax/\expandafter\meaning\string#1}{#2}%
  },
  beamer shortcuts/enable first char </.style={/tikz/ext/beamer shortcuts/@enable first char={<}{\tikzext@beamer@firstchar}},
  beamer shortcuts/enable first char/.value required,
  beamer shortcuts/enable first char/.style={/tikz/ext/beamer shortcuts/@enable first char={#1}{\tikzext@beamer@graphfirstchar}},
  beamer shortcuts/enable handler/.code={\pgfkeysdef{/handlers/.<}{\pgfkeysvalueof{/handlers/.ext_<}##1\pgfeov}},
}
\def\tikzext@temp#1{%
  \pgfkeysdef{/tikz/ext/beamer shortcuts/first char/#1}{\pgfkeyssetvalue{/tikz/ext/beamer shortcuts/first char}{#1}}}
\tikzext@temp{visible}
\tikzext@temp{uncover}
\tikzext@temp{cover}
\tikzext@temp{aobs   visible}
\tikzext@temp{aobs invisible}
\pgfkeyssetvalue{/tikz/ext/beamer shortcuts/first char}{uncover}
% first char syntax: <*overlay*>            → uses "first char"  = *overlay*
%                    <*overlay*>'           → uses "first char"' = *overlay* (i.e. the inverse)
%                    <*overlay*>  {options} → uses \only<*overlay*>{\pgfkeysalso{options}}
%                    <*overlay*>' {options} → uses \alt<*overlay*>{}{\pgfkeysalso{options}} (i.e. the inverse)
\def\tikzext@beamer@firstchar#1{\tikzext@beamer@firstchar@#1\pgf@stop}
\def\tikzext@beamer@firstchar@<#1>{%
  \pgfutil@ifnextchar\pgf@stop{%
    \pgfkeysalso{ext/\pgfkeysvalueof{/tikz/ext/beamer shortcuts/first char}={#1}}%
    \pgfutil@gobble
  }{%
    \pgfutil@ifnextchar'{%
      \tikzext@beamer@firstchar@apo{#1}%
    }{%
      \tikzext@beamer@firstchar@opts{#1}%
    }%
  }%
}
\def\tikzext@beamer@firstchar@opts#1#2\pgf@stop{%
  \beamer@only<#1>{\pgfkeysalso{#2}}%
}
\def\tikzext@beamer@firstchar@apo#1'{%
  \pgfutil@ifnextchar\pgf@stop{%
    \pgfkeysalso{ext/\pgfkeysvalueof{/tikz/ext/beamer shortcuts/first char}'={#1}}%
    \pgfutil@gobble
  }{%
    \tikzext@beamer@firstchar@apo@{#1}%
  }%
}
\def\tikzext@beamer@firstchar@apo@#1#2\pgf@stop{%
  \beamer@alt<#1>{}{\pgfkeysalso{#2}}%
}

\def\tikzext@beamer@graphfirstchar#1{\expandafter\tikzext@beamer@firstchar@\pgfutil@gobble#1\pgf@stop}

% handler .ext_< = *overlay*>            → uses the key w/o options        on *overlay*
%         .ext_< = *overlay*>'           → uses the key w/o options unless on *overlay*
%         .ext_< = *overlay*>  {options} → uses the key  /w options        on *overlay*
%         .ext_< = *overlay*>' {options} → uses the key  /w options unless on *overlay*
\def\tikzext@beamer@handler#1>{%
  \pgfutil@ifnextchar\pgf@stop
    {\beamer@only<#1>{\pgfkeysalso{\pgfkeyscurrentpath}}}%
    {%
      \pgfutil@ifnextchar'%
        {\tikzext@beamer@handler@apo{#1}}%
        {\tikzext@beamer@handler@{#1}}%
    }%
}
\def\tikzext@beamer@handler@#1#2\pgf@stop{%
  \beamer@only<#1>{\pgfkeysalso{\pgfkeyscurrentpath={#2}}}%
}
\def\tikzext@beamer@handler@apo#1'{%
  \pgfutil@ifnextchar\pgf@stop{%
    \beamer@alt<#1>{}{\pgfkeysalso{\pgfkeyscurrentpath}}%
    \pgfutil@gobble
  }{%
    \tikzext@beamer@handler@apo@opts{#1}%
  }%
}
\def\tikzext@beamer@handler@apo@opts#1#2\pgf@stop{%
  \beamer@alt<#1>{}{\pgfkeysalso{\pgfkeyscurrentpath={#2}}}%
}

% ext/ keys need special handling in graphs library
\pgfqkeys{/tikz/graphs}{
  ext/uncover/.style={
    /tikz/graphs/@edges styling/.append={,/tikz/ext/uncover={#1}},
    /tikz/ext/uncover={#1}},
  ext/visible/.style={
    /tikz/graphs/@edges styling/.append={,/tikz/ext/visible={#1}},
    /tikz/ext/visible={#1}},
  ext/cover/.style={
    /tikz/graphs/@edges styling/.append={,/tikz/ext/cover={#1}}
    /tikz/ext/cover={#1}},
  ext/invisible/.style={
    /tikz/graphs/@edges styling/.append={,/tikz/ext/invisible={#1}}
    /tikz/ext/invisible={#1}}
}
\ifcsname tikz@library@graphs@loaded\endcsname
  \tikzgraphsset{
    new ->/.code n args={4}{%
      \path [arrows=->,every new ->/.try] (#1\tikzgraphleftanchor)
                              edge[#3] #4 (#2\tikzgraphrightanchor);},
    new --/.code n args={4}{%
        \path [arrows=-,every new --/.try] (#1\tikzgraphleftanchor)
                              edge[#3] #4 (#2\tikzgraphrightanchor);},
    new <->/.code n args={4}{%
      \path [arrows=<->,every new <->/.try] (#1\tikzgraphleftanchor)
                                edge[#3] #4 (#2\tikzgraphrightanchor);
      \typeout{%
        \noexpand\path [arrows=<->,every new <->/.try] (#1\tikzgraphleftanchor)
                                          edge[#3] #4 (#2\tikzgraphrightanchor);}},
    new -!-/.code n args={4}{},
    new <-/.code n args={4}{%
      \path [arrows=<-,every new <-/.try] (#1\tikzgraphleftanchor)
                              edge[#3] #4 (#2\tikzgraphrightanchor);}}%
\fi