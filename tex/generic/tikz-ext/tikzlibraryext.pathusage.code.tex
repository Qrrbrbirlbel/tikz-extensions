% Copyright 2023 by Qrrbrbirlbel
%
% This file may be distributed and/or modified
%
% 1. under the LaTeX Project Public License and/or
% 2. under the GNU Free Documentation License.
%
\usepgflibrary{ext.pathusage}
\newif\iftikz@mode@undraw     % don't draw but add to bb
\newif\iftikz@mode@drawundraw % draw but don't add to bb
\tikzset{
  undraw/.is choice,
  undraw/true/.code=%
    \tikz@addmode{%
      \tikz@mode@drawfalse
      \tikz@mode@undrawtrue
      \tikz@mode@drawundrawfalse
    },
  undraw/false/.code=%
    \tikz@addmode{%
      \tikz@mode@undrawfalse
    },
  undraw/.default=true,
  draw undraw/.default=,
  draw undraw/.code=%
    \edef\tikz@temp{#1}%
    \ifx\tikz@temp\tikz@nonetext
      \tikz@addmode{%
        \tikz@mode@drawundrawfalse
      }%
    \else
      \ifx\tikz@temp\pgfutil@empty
      \else
        \tikz@addoption{\pgfsetstrokecolor{#1}}%
        \def\tikz@strokecolor{#1}%
      \fi
      \tikz@addmode{%
        \tikz@mode@drawtrue
        \tikz@mode@undrawfalse
        \tikz@mode@drawundrawtrue
      }%
    \fi,
  draw/.default=,
  draw/.code=%
    \edef\tikz@temp{#1}%
    \ifx\tikz@temp\tikz@nonetext%
      \tikz@addmode{\tikz@mode@drawfalse}%
    \else%
      \ifx\tikz@temp\pgfutil@empty%
      \else%
        \tikz@addoption{\pgfsetstrokecolor{#1}}%
        \def\tikz@strokecolor{#1}%
      \fi%
      \tikz@addmode{%
        \tikz@mode@drawtrue
        \tikz@mode@undrawfalse
      }%
    \fi%,
}

\def\tikz@finish{%
  % Rendering pipeline
  %
  % Step 1: The path background box
  %
  \box\tikz@figbox@bg%
  %
  % Step 2: Decorate path
  %
  \iftikz@decoratepath%
    \tikz@lib@dec@decorate@path%
  \fi%
  %
  % Step 3: Preactions
  %
  \pgfsyssoftpath@getcurrentpath\tikz@actions@path%
  \edef\tikz@restorepathsize{%
    \global\pgf@pathmaxx=\the\pgf@pathmaxx%
    \global\pgf@pathmaxy=\the\pgf@pathmaxy%
    \global\pgf@pathminx=\the\pgf@pathminx%
    \global\pgf@pathminy=\the\pgf@pathminy%
  }%
  \tikz@preactions%
  %
  % Step 4: Reset modes
  %
  \let\tikz@path@picture=\pgfutil@empty%
  \tikz@mode@fillfalse%
  \tikz@mode@drawfalse%
  \tikz@mode@undrawfalse     % ←
  \tikz@mode@drawundrawfalse % ←
  \tikz@mode@doublefalse%
  \tikz@mode@clipfalse%
  \tikz@mode@boundaryfalse%
  \tikz@mode@fade@pathfalse%
  \tikz@mode@fade@scopefalse%
  \edef\tikz@pathextend{%
    {\noexpand\pgfqpoint{\the\pgf@pathminx}{\the\pgf@pathminy}}%
    {\noexpand\pgfqpoint{\the\pgf@pathmaxx}{\the\pgf@pathmaxy}}%
  }%
  \let\tikz@beamermode\pgfutil@empty % ← for ext.beamer
  \tikz@mode% installs the mode settings
  \tikz@beamermode                   % ← for ext.beamer
  % Path fading counts as an option:
  \iftikz@mode@fade@path%
    \tikz@addoption{%
      \iftikz@fade@adjust%
        \iftikz@mode@draw%
          \pgfsetfadingforcurrentpathstroked{\tikz@path@fading}{\tikz@do@fade@transform}%
        \else%
          \pgfsetfadingforcurrentpath{\tikz@path@fading}{\tikz@do@fade@transform}%
        \fi%
      \else%
        \pgfsetfading{\tikz@path@fading}{\tikz@do@fade@transform}%
      \fi%
      \tikz@mode@fade@pathfalse% no more fading...
    }%
  \fi%
  %
  % Step 5: Install scope fading
  %
  \iftikz@mode@fade@scope%
    \iftikz@fade@adjust%
      \iftikz@mode@draw%
        \pgfsetfadingforcurrentpathstroked{\tikz@scope@fading}{\tikz@do@fade@transform}%
      \else%
        \pgfsetfadingforcurrentpath{\tikz@scope@fading}{\tikz@do@fade@transform}%
      \fi%
    \else%
      \pgfsetfading{\tikz@scope@fading}{\tikz@do@fade@transform}%
    \fi%
    \tikz@mode@fade@scopefalse%
  \fi%
  %
  % Step 5': Setup options
  %
  \ifx\tikz@options\pgfutil@empty%
  \else%
    \pgfsys@beginscope%
      \let\pgfscope@stroke@color=\pgf@strokecolor@global%
      \let\pgfscope@fill@color=\pgf@fillcolor@global%
      \begingroup%
        \tikz@options%
  \fi%
  \tikz@do@rdf@pre@options%
  %
  % Step 5'': Setup animations
  %
  \tikz@is@nodefalse%
  \tikz@call@id@hook%
  \iftikz@mode@clip\else%
    \pgfidscope%
      \tikz@do@rdf@post@options%
      \begingroup%
  \fi% open an animation scope here, unless clipping is done
  %
  % Step 6: Do a fill if shade or a path picture follows.
  %
  \iftikz@mode@fill%
    \iftikz@mode@shade%
      \pgfsyssoftpath@getcurrentpath\tikz@temppath
      \pgfprocessround{\tikz@temppath}{\tikz@temppath}% change the path
      \pgfsyssoftpath@setcurrentpath\tikz@temppath%
      \pgfsyssoftpath@invokecurrentpath%
      \pgfpushtype%
      \pgfusetype{.path fill}%
      \pgfsys@fill%
      \pgfpoptype%
      \tikz@mode@fillfalse% no more filling...
    \else%
      \ifx\tikz@path@picture\pgfutil@empty%
      \else%
        \pgfsyssoftpath@getcurrentpath\tikz@temppath
        \pgfprocessround{\tikz@temppath}{\tikz@temppath}% change the path
        \pgfsyssoftpath@setcurrentpath\tikz@temppath%
        \pgfsyssoftpath@invokecurrentpath%
        \pgfpushtype%
        \pgfusetype{.path fill}%
        \pgfsys@fill%
        \pgfpoptype%
        \tikz@mode@fillfalse% no more filling...
      \fi%
    \fi%
  \fi%
  %
  % Step 7: Do a shade if necessary.
  %
  \iftikz@mode@shade%
    \pgfsyssoftpath@getcurrentpath\tikz@temppath
    \pgfprocessround{\tikz@temppath}{\tikz@temppath}% change the path
    \pgfsyssoftpath@setcurrentpath\tikz@temppath%
    \pgfpushtype%
    \pgfusetype{.path shade}%
    \pgfshadepath{\tikz@shading}{\tikz@shade@angle}%
    \pgfpoptype%
    \tikz@mode@shadefalse% no more shading...
  \fi%
  %
  % Step 8: Do a path picture if necessary.
  %
  \ifx\tikz@path@picture\pgfutil@empty%
  \else%
    \begingroup%
      \pgfusetype{.path picture}%
      \pgfidscope%
      \pgfsys@beginscope%
        \let\tikz@id@name\pgfutil@empty%
        \pgfclearid%
        \pgfsyssoftpath@getcurrentpath\tikz@temppath
        \pgfprocessround{\tikz@temppath}{\tikz@temppath}% change the path
        \pgfsyssoftpath@setcurrentpath\tikz@temppath%
        \pgfsyssoftpath@invokecurrentpath%
        \pgfsys@clipnext%
        \pgfsys@discardpath%
        \pgf@relevantforpicturesizefalse%
        \expandafter\def\csname pgf@sh@ns@path picture bounding box\endcsname{rectangle}
        \expandafter\edef\csname pgf@sh@np@path picture bounding box\endcsname{%
          \def\noexpand\southwest{\noexpand\pgfqpoint{\the\pgf@pathminx}{\the\pgf@pathminy}}%
          \def\noexpand\northeast{\noexpand\pgfqpoint{\the\pgf@pathmaxx}{\the\pgf@pathmaxy}}%
        }
        \expandafter\def\csname pgf@sh@nt@path picture bounding box\endcsname{{1}{0}{0}{1}{0pt}{0pt}}
        \expandafter\def\csname pgf@sh@pi@path picture bounding box\endcsname{\pgfpictureid}
        \pgfinterruptpath%
          \tikz@path@picture%
        \endpgfinterruptpath%
      \pgfsys@endscope%
      \endpgfidscope%
    \endgroup%
    \let\tikz@path@picture=\pgfutil@empty%
  \fi%
  %
  % Step 9: Double stroke, if necessary
  %
  \iftikz@mode@draw%
    \iftikz@mode@double%
      % Change line width
      \begingroup%
        \pgfsys@beginscope%
          \tikz@double@setup%
    \fi%
  \fi%
  %
  % Step 10: Do stroke/fill/clip as needed
  %
  \pgfpushtype%
  \edef\tikz@temp{\noexpand\pgfusepath{%
    \iftikz@mode@fill fill,\fi%
    \iftikz@mode@undraw undraw,\fi%
    \iftikz@mode@draw draw\iftikz@mode@drawundraw\space undraw\fi,\fi%
    \iftikz@mode@clip clip\fi%
    }}%
  \pgfusetype{.path}%
  \tikz@temp%
  \pgfpoptype%
  \tikz@mode@fillfalse% no more filling
  %
  % Step 11: Double stroke, if necessary
  %
  \iftikz@mode@draw%
    \iftikz@mode@double%
        \pgfsys@endscope%
      \endgroup%
    \fi%
  \fi%
  \tikz@mode@drawfalse% no more stroking
  %
  % Step 12: Postactions
  %
  \tikz@postactions%
  %
  % Step 13: Add labels and nodes
  %
  \box\tikz@figbox%
  %
  % Step 14: Close animations
  %
  \iftikz@mode@clip\else\endgroup\endpgfidscope\fi%
  %
  % Step 14: Close option brace
  %
  \ifx\tikz@options\pgfutil@empty%
  \else%
      \endgroup%
      \global\let\pgf@strokecolor@global=\pgfscope@stroke@color%
      \global\let\pgf@fillcolor@global=\pgfscope@fill@color%
    \pgfsys@endscope%
    \iftikz@mode@clip%
      \tikzerror{Extra options not allowed for clipping path command.}%
    \fi%
  \fi%
  \iftikz@mode@clip%
    \aftergroup\pgf@relevantforpicturesizefalse%
  \fi%
  \iftikz@mode@boundary%
    \aftergroup\pgf@relevantforpicturesizefalse%
  \fi%
  \endgroup%
  \global\pgflinewidth=\tikzscope@linewidth%
  \tikz@path@do@at@end%
}%